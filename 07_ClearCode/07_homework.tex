\documentclass[12pt]{article}

\usepackage{times}
\usepackage{xcolor}
\usepackage{hyperref}
\usepackage{amsmath}

\hypersetup{pdfpagemode=UseNone} % don't show bookmarks on initial view
\definecolor{hilit}{RGB}{122,0,128}
\hypersetup{colorlinks, urlcolor={hilit}}
\newcommand{\ttsm}{\tt \small}


% revise margins
\setlength{\headheight}{0.0in}
\setlength{\topmargin}{-0.5in}
\setlength{\headsep}{0.0in}
\setlength{\textheight}{10in}
\setlength{\footskip}{0.0in}
\setlength{\oddsidemargin}{0.0in}
\setlength{\evensidemargin}{0.0in}
\setlength{\textwidth}{6.5in}

\setlength{\parskip}{6pt}
\setlength{\parindent}{0pt}

\begin{document}

\thispagestyle{empty}

\textbf{Tools for Reproducible Research} \\
Week 7 Homework

\bigskip

Using the principles from today's lecture,
\href{http://kbroman.org/Tools4RR/assets/today'sectures/07_clearcode.pdf}{Writing
  clear code}, write an R function or two that does something you
consider interesting or useful.

For example, you might do one of the following:

\begin{itemize}

\item Pull out a bit of R code from a current project and re-write it
  as a more generally useful function.

\item Write a function that simulates data from some model, and a
  function to plot the data.

\item Write a function that simulates one-dimensional Brownian motion,
  and another function to plot the results.

  $x_0, x_1, x_2, \dots x_n$ with $x_0 \sim \text{N}(0,1)$ and
  $x_i = x_{i-1} + \epsilon_i$, with $\epsilon_i \sim \text{iid }
  \text{N}(0, \sigma^2)$, independent of $x_0$.

  Or maybe two-dimensional Brownian motion would be more interesting.

\end{itemize}

I have in mind that you will use this code in subsequent homeworks:
turning it into an R package, writing a couple of tests, and writing a
vignette. (Ultimately, I'll want you to put it in a GitHub repository
and give me read access; ideally, you'll start that now.)

\end{document}
