\documentclass[12pt,t]{beamer}
\usepackage{graphicx}
\setbeameroption{hide notes}
\setbeamertemplate{note page}[plain]
\usepackage{listings}

% set up listing environment
\lstset{language=bash,
        basicstyle=\ttfamily\scriptsize,
        frame=single,
        commentstyle=,
        backgroundcolor=\color{darkgray},
        showspaces=false,
        showstringspaces=false
        }

% get rid of junk
\usetheme{default}
\beamertemplatenavigationsymbolsempty
\hypersetup{pdfpagemode=UseNone} % don't show bookmarks on initial view


% font
\usepackage{fontspec}
\setsansfont
  [ ExternalLocation = ../fonts/ ,
    UprightFont = *-regular , 
    BoldFont = *-bold ,
    ItalicFont = *-italic ,
    BoldItalicFont = *-bolditalic ]{texgyreheros}
\setbeamerfont{note page}{family*=pplx,size=\footnotesize} % Palatino for notes
% "TeX Gyre Heros can be used as a replacement for Helvetica"
% I've placed them in ../fonts/; alternatively you can install them
% permanently on your system as follows:
%     Download http://www.gust.org.pl/projects/e-foundry/tex-gyre/heros/qhv2.004otf.zip
%     In Unix, unzip it into ~/.fonts
%     In Mac, unzip it, double-click the .otf files, and install using "FontBook"

% named colors
\definecolor{offwhite}{RGB}{249,242,215}
\definecolor{foreground}{RGB}{255,255,255}
\definecolor{background}{RGB}{24,24,24}
\definecolor{title}{RGB}{107,174,214}
\definecolor{gray}{RGB}{155,155,155}
\definecolor{subtitle}{RGB}{102,255,204}
\definecolor{hilit}{RGB}{102,255,204}
\definecolor{vhilit}{RGB}{255,111,207}
\definecolor{nhilit}{RGB}{128,0,128}  % hilit color in notes
\definecolor{nvhilit}{RGB}{255,0,128} % vhilit for notes
\definecolor{lolit}{RGB}{155,155,155}

\newcommand{\hilit}{\color{hilit}}
\newcommand{\vhilit}{\color{vhilit}}
\newcommand{\nhilit}{\color{nhilit}}
\newcommand{\nvhilit}{\color{nvhilit}}
\newcommand{\lolit}{\color{lolit}}

% use those colors
\setbeamercolor{titlelike}{fg=title}
\setbeamercolor{subtitle}{fg=subtitle}
\setbeamercolor{institute}{fg=gray}
\setbeamercolor{normal text}{fg=foreground,bg=background}
\setbeamercolor{item}{fg=foreground} % color of bullets
\setbeamercolor{subitem}{fg=gray}
\setbeamercolor{itemize/enumerate subbody}{fg=gray}
\setbeamertemplate{itemize subitem}{{\textendash}}
\setbeamerfont{itemize/enumerate subbody}{size=\footnotesize}
\setbeamerfont{itemize/enumerate subitem}{size=\footnotesize}

% page number
\setbeamertemplate{footline}{%
    \raisebox{5pt}{\makebox[\paperwidth]{\hfill\makebox[20pt]{\lolit
          \scriptsize\insertframenumber}}}\hspace*{5pt}}

% add a bit of space at the top of the notes page
\addtobeamertemplate{note page}{\setlength{\parskip}{12pt}}

% default link color
\hypersetup{colorlinks, urlcolor={hilit}}

% a few macros
\newcommand{\bi}{\begin{itemize}}
\newcommand{\bbi}{\vspace{24pt} \begin{itemize} \itemsep8pt}
\newcommand{\ei}{\end{itemize}}
\newcommand{\ig}{\includegraphics}
\newcommand{\subt}[1]{{\footnotesize \color{subtitle} {#1}}}
\newcommand{\ttsm}{\tt \small}
\newcommand{\ttfn}{\tt \footnotesize}
\newcommand{\figh}[2]{\centerline{\includegraphics[height=#2\textheight]{#1}}}
\newcommand{\figw}[2]{\centerline{\includegraphics[width=#2\textwidth]{#1}}}


%%%%%%%%%%%%%%%%%%%%%%%%%%%%%%%%%%%%%%%%%%%%%%%%%%%%%%%%%%%%%%%%%%%%%%
% end of header
%%%%%%%%%%%%%%%%%%%%%%%%%%%%%%%%%%%%%%%%%%%%%%%%%%%%%%%%%%%%%%%%%%%%%%

\title{KnitR + \LaTeX\/ $\rightarrow$ paper}
\subtitle{Tools for Reproducible Research}
\author{\href{http://www.biostat.wisc.edu/~kbroman}{Karl Broman}}
\institute{Biostatistics \& Medical Informatics, UW{\textendash}Madison}
\date{\href{http://www.biostat.wisc.edu/~kbroman}{\tt \scriptsize \color{foreground} biostat.wisc.edu/{\textasciitilde}kbroman}
\\[-4pt]
\href{http://github.com/kbroman}{\tt \scriptsize \color{foreground} github.com/kbroman}
\\[-4pt]
\href{https://twitter.com/kwbroman}{\tt \scriptsize \color{foreground} @kwbroman}
\\[-4pt]
{\scriptsize Course web: \href{http://bit.ly/tools4rr}{\tt bit.ly/tools4rr}}
}

\begin{document}

{
\setbeamertemplate{footline}{} % no page number here
\frame{
  \titlepage

\note{
  This lecture is about how to create reproducible manuscripts, for
  journal articles. KnitR with R Markdown is great for informal
  reports. KnitR with AsciiDoc is great for somewhat fancier
  reports. There are a number of efforts, especially with Pandoc, to
  use R Markdown for journal articles. But if you want fine control
  over the appearance of a document, it's hard to beat \LaTeX, and so
  I'm just going to focus on that.

  I can't hope to explain \LaTeX\/ properly in just this one
  lecture. My goals are to give the general gist, indicate resources
  and options, and show how to use KnitR with \LaTeX.
}
} }


\begin{frame}[c,fragile]{\LaTeX}

\begin{lstlisting}
\documentclass[12pt]{article}

\usepackage{graphicx}

\title{An example document}
\author{Karl Broman}

\begin{document}

\maketitle
\thispagestyle{empty}

\section{A section}

This is a simple example of a \LaTeX\/ document for an article.
Here's some in-line math: $y = \beta_0 + \beta_1 x + \epsilon$.

And here's a display equation:

$$ \hat{\beta} = (X'X)^{-1} X'y $$

\end{document}
\end{lstlisting}

\note{
  \LaTeX\/ is like html or Markdown: plain text with special codes to
  indicate how things are to appear.

  A \LaTeX\/ document always starts with {\tt {\textbackslash}documentclass}, then a
  bunch of overall controlling information. The actual document is
  between {\tt {\textbackslash}begin\{document\}} and {\tt {\textbackslash}end\{document\}}.

  {\tt {\textbackslash}usepackage\{\}} is like {\tt library()} in R.

  Ideally, you focus on {\nhilit semantics} rather than {\nhilit
  style}: define the {\tt {\textbackslash}title\{\}} and {\tt {\textbackslash}author\{\}} and use
  {\tt {\textbackslash}maketitle} to have them included in the document, and
  indicate sections and subsections with {\tt {\textbackslash}section\{\}} and
  {\tt {\textbackslash}subsection\{\}}.

  For some reason, {\tt {\textbackslash}thispagestyle\{empty\}} (``don't
  show page number on this page'') needs
  to be placed {\nhilit after} {\tt {\textbackslash}maketitle}.

  A key feature of \LaTeX\/ is the mathematics typesetting. There's no
  better system. And your \LaTeX\/ skills can be immediately
  transferred to your Markdown documents.
}
\end{frame}


\begin{frame}[c,fragile]{What I actually do}

\begin{lstlisting}
\documentclass[12pt]{article}

\setlength{\headheight}{10pt}
\setlength{\headsep}{15pt}
\setlength{\topmargin}{-25pt}
\setlength{\topskip}{0in}
\setlength{\textheight}{8.7in}
\setlength{\footskip}{0.3in}
\setlength{\oddsidemargin}{0.0in}
\setlength{\evensidemargin}{0.0in}
\setlength{\textwidth}{6.5in}

\begin{document}
\begin{center}
\textbf{\large An example document}

\vspace{10mm}
Karl Broman
\end{center}

\vspace{30mm}
\textbf{\sffamily A section}
\end{lstlisting}

\note{
  In reality, for a paper, I don't use {\tt {\textbackslash}maketile}
  or {\tt {\textbackslash}section}, but rather just muck about,
  hard-coding the placement of things.

  But mine is not the recommended approach. If, for some reason, you
  need to change the style, it's easier if your document is defined in
  terms of {\nhilit semantics}.
}
\end{frame}


\begin{frame}{Why \LaTeX\/?}

\bbi
\item Fine control of document appearance
\item Transparency of how that was achieved
\item Version control (diff/merge)
\item Typesetting equations
\item Markdown's not quite ready, or sufficiently rich
\ei


\note{
  It's {\nhilit a lot} of work to learn \LaTeX, so we need to be clear
  about why we'd want to devote the effort to it.

  For reproducible research, we need some sort of code-based document
  system (i.e., {\nvhilit not Word!}), and \LaTeX\/ gives you the most
  fine-grained control, if you need it. Ultimately, I hope, Markdown
  will be sufficient, but for now, we often need \LaTeX\/.

  The code-based control makes what you're trying to do
  transparent.  And you should treat \LaTeX\/ like code: write clearly
  and simply, and comment the tricky bits.

  This sort of document also has the advantage of easy treatment of
  {\tt diff} and {\tt merge} in a version control system like git.

  The real power of \LaTeX\/ is in the typesetting of mathematical
  equations. And what you learn on that aspect can be transferred to
  your Markdown documents, using MathJax.
}
\end{frame}


\begin{frame}[c]{}

\vspace{10mm}

\centerline{\Large simple \quad $\longleftrightarrow$ \quad flexible}

\vspace{10mm}

\onslide<2->{\centerline{\tt \scriptsize \lolit {\textbackslash}centerline\{{\textbackslash}Large simple
{\textbackslash}quad \${\textbackslash}longleftrightarrow\$ {\textbackslash}quad flexible\}}}

\note{
  \LaTeX\/ sits at the right of the simple-to-flexible spectrum.
}
\end{frame}



\begin{frame}[c]{}

\centering
Modify your desires to match the defaults.

\vspace{36pt}

Focus your compulsive behavior on things that matter.

\note{
  I've said this before, but I like to repeat it.

  Focus on the text and the figures before worrying too much about
  fine details of how they appear on the page.

  And consider which is more important: a manuscript, web page, blog,
  grant, course slides, course handout, report to collaborator,
  scientific poster.

  You can spend a ton of time trying to get things to look just
  right. Ideally, you spend that time trying to construct a general
  solution.  Or you can modify your desires to more closely match what
  you get without any effort.
}
\end{frame}



\begin{frame}<handout:0>[c,fragile]{KnitR + \LaTeX $\rightarrow$ Rnw}

\begin{lstlisting}
\documentclass[12pt]{article}

\title{An example Rnw document}
\author{Karl Broman}

\begin{document}
\maketitle

<<load_library, echo=FALSE, results="hide">>=
library(broman) # used for myround()
@

<<example_chunk>>=
x <- rnorm(100)
y <- 5*x + rnorm(100)
lm.out <- lm(y ~ x)
plot(x,y)
abline(lm.out$coef)
@

The estimated slope is \Sexpr{myround(lm.out$coef[2], 1)}.
\end{document}
\end{lstlisting}
\end{frame}



\begin{frame}[c,fragile]{KnitR + \LaTeX $\rightarrow$ Rnw}
\addtocounter{framenumber}{-1}

\begin{lstlisting}
\documentclass[12pt]{article}

\title{An example Rnw document}
\author{Karl Broman}

\begin{document}
\maketitle

<<load_library, echo=FALSE, results="hide">>=
library(broman) # used for myround()
@

<<example_chunk, out.width="0.8\\textwidth">>=
x <- rnorm(100)
y <- 5*x + rnorm(100)
lm.out <- lm(y ~ x)
plot(x,y)
abline(lm.out$coef)
@

The estimated slope is \Sexpr{myround(lm.out$coef[2], 1)}.
\end{document}
\end{lstlisting}

\note{
  KnitR works well with LaTeX.

  Most of what you learned about KnitR with RMarkdown transfers
  directly to working with LaTeX.

  The main difference is the way in which code chunks are
  indicated. You use {\tt <<>>=} and {\tt @} for chunks, and
  {\tt {\textbackslash}Sexpr\{\}} for in-line code.

  KnitR basically does a search-and-replace for code
  chunks. Different patterns will be easier, depending on the nature
  of the surrounding code.

  The chunk options are the same. Here, I used
  {\tt out.width="0.8{\textbackslash\textbackslash}textwidth"}
  to make the figure appear as 80\% of the width of the page.

  {\tt out.width} and {\tt out.height} need units as in \LaTeX\/ (built into
  {\tt {\textbackslash\textbackslash}textwidth}; otherwise {\tt "in"}
  or {\tt "cm"} or {\tt "pt"} or whatever).

  {\tt fig.width} and {\tt fig.height} are as in R, with implied units.
}
\end{frame}

\begin{frame}{LyX}

\vspace{20pt}

\figh{Figs/lyx.png}{0.7}

\vspace{20pt}

\hfill \href{http://www.lyx.org/}{lyx.org}

\note{
  I create \LaTeX\/ documents in emacs. If you want something WYSIWYG,
  consider LyX. KnitR is built-in, and Yihui Xie strongly endorses
  it.
}
\end{frame}


\begin{frame}{Also}

\bbi
\item \href{http://writelatex.com}{WriteLaTeX}
\item \href{http://authorea.com}{Authorea}
\item \href{http://sharelatex.com}{ShareLaTeX}
\item \href{http://verbosus.com}{Verbosus}
\ei

\note{
  There are a bunch of online tools for creating LaTeX documents,
  collaboratively.

  I have no experience with these, but I've heard good things about
  WriteLaTeX.
}
\end{frame}



\begin{frame}{Getting help}

\bbi
\item Google
\item \href{tex.stackexchange.com}{tex.stackexchange.com}
\item Ask a friend
\item Look at others' documents
\item Resign yourself to something less-than-ideal
\ei

\note{
  There is {\nhilit a ton} of online information about \LaTeX. Start
  with google. It's highly unlikely that you have a completely unique
  question or problem.

  My last point here is basically that one way to help yourself is by
  learning to let things go.
}
\end{frame}



\begin{frame}[c,fragile]{Figure captions and floats}
\addtocounter{framenumber}{-1}

\begin{lstlisting}
<<fig_with_caption, fig.cap="Scatterplot of $y$ vs $x$">>=
x <- rnorm(100)
y <- 5*x + rnorm(100)
lm.out <- lm(y ~ x)
plot(x,y)
abline(lm.out$coef)
@
\end{lstlisting}

\bigskip

\begin{lstlisting}
\begin{figure}[]
\includegraphics{figure/fig_with_caption}

\caption{Scatterplot of $y$ vs $x$\label{fig:fig_with_caption}}
\end{figure}
\end{lstlisting}


\note{
  If you use the chunk option {\tt fig.cap}, the figure will get a
  caption.

  But it will also be embedded within a {\tt figure} ``environment.''
  (That is, between {\tt {\textbackslash}begin\{figure\}} and
  {\tt {\textbackslash}end\{figure\}}.)

  This makes it a ``float.'' \LaTeX\/ decides where it's going to be
  placed. The placement of floats is {\nvhilit the biggest pain} in
  using \LaTeX.

  The figure also gets a label, from the chunk name. (The
  {\tt {\textbackslash}label\{\}} bit.) This allows you to
  cross-reference the figure, to have the figure number determined
  automatically.

  The cross-reference would be with {\tt {\textbackslash}ref\{fig:fig\_with\_caption\}}.
}
\end{frame}




\begin{frame}{Tables in \LaTeX}


\note{
}
\end{frame}



\begin{frame}{xtable}

\note{
}
\end{frame}


\begin{frame}{BibTeX for bibliographies}

\note{
}
\end{frame}


\end{document}
