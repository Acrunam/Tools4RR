\documentclass[12pt]{article}

\usepackage{times}
\usepackage{xcolor}
\usepackage{hyperref}

\hypersetup{pdfpagemode=UseNone} % don't show bookmarks on initial view
\definecolor{hilit}{RGB}{122,0,128}
\hypersetup{colorlinks, urlcolor={hilit}}
\newcommand{\ttsm}{\tt \small}


% revise margins
\setlength{\headheight}{0.0in}
\setlength{\topmargin}{-0.5in}
\setlength{\headsep}{0.0in}
\setlength{\textheight}{10in}
\setlength{\footskip}{0.0in}
\setlength{\oddsidemargin}{0.0in}
\setlength{\evensidemargin}{0.0in}
\setlength{\textwidth}{6.5in}

\setlength{\parskip}{6pt}
\setlength{\parindent}{0pt}

\begin{document}

\thispagestyle{empty}

\textbf{Tools for Reproducible Research} \\
Week 1 Homework

\bigskip

\begin{enumerate}
%\item Visit the following doodle poll and indicate your preferred
%  times for office hours.
%
%  \href{http://doodle.com/rrw8skdzadnymide}{\ttsm bit.ly/BromanSpring2015OfficeHours}

\item Install a bunch of software on your laptop.

  \begin{enumerate}
  \item Windows

    \begin{itemize}
    \item \href{http://cran.rstudio.com/bin/windows/base/}{R}
    \item \href{http://cran.rstudio.com/bin/windows/Rtools/Rtools32.exe}{Rtools}
      (includes \href{http://www.gnu.org/software/make/}{\ttsm make})
    \item \href{http://www.rstudio.com/products/rstudio/download/}{RStudio}
    \item \href{http://msysgit.github.io/}{Git Bash}
    \item An editor, such as \href{http://notepad-plus-plus.org/download}{Notepad++}
    \item \href{http://johnmacfarlane.net/pandoc/installing.html}{Pandoc}
    \item \href{http://miktex.org/download}{MikTeX} (for LaTeX); add
      it to your {\ttsm PATH} (for example, see
      \href{http://www.howtogeek.com/118594/how-to-edit-your-system-path-for-easy-command-line-access/}{this page}).
    \end{itemize}


  \item Mac

    \begin{itemize}
    \item \href{http://cran.rstudio.com/bin/macosx/}{R}
    \item \href{http://www.rstudio.com/products/rstudio/download}{RStudio}
    \item Xcode command line tools (see
      \href{http://railsapps.github.io/xcode-command-line-tools.html}{this
        page})
    \item \href{http://git-scm.com/download/mac}{git} (via
      \href{http://brew.sh/}{Homebrew}: {\ttsm brew install git})
    \item Perhaps \href{http://iterm2.com/}{iTerm2}
    \item An editor, such as
      \href{http://www.sublimetext.com/}{Sublime Text}
    \item \href{http://johnmacfarlane.net/pandoc/installing.html}{Pandoc}
        (via \href{http://brew.sh/}{Homebrew}: {\ttsm brew install pandoc})
    \item \href{https://tug.org/mactex/}{MacTeX} or
      \href{http://www.tug.org/mactex/morepackages.html}{BasicTex} (for LaTeX)
    \end{itemize}

  \end{enumerate}

\item Download the course repository.

  \begin{enumerate}
  \item Open a terminal window.
  \item Type {\ttsm git clone git://github.com/kbroman/Tools4RR}
  \end{enumerate}

\item Try out GNU Make.

  \begin{enumerate}
  \item Change to the Lec 1 examples directory: \\
    {\ttsm cd Tools4RR/01\_Intro/Examples}
  \item Create a subdirectory for the figures: \\
    {\ttsm mkdir Figs}
  \item Use make to create {\ttsm Figs/fig1.pdf}: \\
    {\ttsm make Figs/fig1.pdf}
  \item Make the whole paper:\\
    {\ttsm make}
  \item Use the {\ttsm Makefile\_fancy} file to remove all the
    pdfs:\\
    {\ttsm make -f Makefile\_fancy clean}
  \end{enumerate}

\end{enumerate}


\end{document}
