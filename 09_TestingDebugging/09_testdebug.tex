\documentclass[12pt,t]{beamer}
\usepackage{graphicx}
\setbeameroption{hide notes}
\setbeamertemplate{note page}[plain]
\usepackage{listings}

% set up listing environment
\lstset{language=bash,
        basicstyle=\ttfamily\scriptsize,
        frame=single,
        commentstyle=,
        backgroundcolor=\color{darkgray},
        showspaces=false,
        showstringspaces=false
        }

% get rid of junk
\usetheme{default}
\beamertemplatenavigationsymbolsempty
\hypersetup{pdfpagemode=UseNone} % don't show bookmarks on initial view


% font
\usepackage{fontspec}
\setsansfont
  [ ExternalLocation = ../fonts/ ,
    UprightFont = *-regular , 
    BoldFont = *-bold ,
    ItalicFont = *-italic ,
    BoldItalicFont = *-bolditalic ]{texgyreheros}
\setbeamerfont{note page}{family*=pplx,size=\footnotesize} % Palatino for notes
% "TeX Gyre Heros can be used as a replacement for Helvetica"
% I've placed them in ../fonts/; alternatively you can install them
% permanently on your system as follows:
%     Download http://www.gust.org.pl/projects/e-foundry/tex-gyre/heros/qhv2.004otf.zip
%     In Unix, unzip it into ~/.fonts
%     In Mac, unzip it, double-click the .otf files, and install using "FontBook"

% named colors
\definecolor{offwhite}{RGB}{249,242,215}
\definecolor{foreground}{RGB}{255,255,255}
\definecolor{background}{RGB}{24,24,24}
\definecolor{title}{RGB}{107,174,214}
\definecolor{gray}{RGB}{155,155,155}
\definecolor{subtitle}{RGB}{102,255,204}
\definecolor{hilit}{RGB}{102,255,204}
\definecolor{vhilit}{RGB}{255,111,207}
\definecolor{nhilit}{RGB}{128,0,128}  % hilit color in notes
\definecolor{nvhilit}{RGB}{255,0,128} % vhilit for notes
\definecolor{lolit}{RGB}{155,155,155}

\newcommand{\hilit}{\color{hilit}}
\newcommand{\vhilit}{\color{vhilit}}
\newcommand{\nhilit}{\color{nhilit}}
\newcommand{\nvhilit}{\color{nvhilit}}
\newcommand{\lolit}{\color{lolit}}

% use those colors
\setbeamercolor{titlelike}{fg=title}
\setbeamercolor{subtitle}{fg=subtitle}
\setbeamercolor{institute}{fg=gray}
\setbeamercolor{normal text}{fg=foreground,bg=background}
\setbeamercolor{item}{fg=foreground} % color of bullets
\setbeamercolor{subitem}{fg=gray}
\setbeamercolor{itemize/enumerate subbody}{fg=gray}
\setbeamertemplate{itemize subitem}{{\textendash}}
\setbeamerfont{itemize/enumerate subbody}{size=\footnotesize}
\setbeamerfont{itemize/enumerate subitem}{size=\footnotesize}

% page number
\setbeamertemplate{footline}{%
    \raisebox{5pt}{\makebox[\paperwidth]{\hfill\makebox[20pt]{\lolit
          \scriptsize\insertframenumber}}}\hspace*{5pt}}

% add a bit of space at the top of the notes page
\addtobeamertemplate{note page}{\setlength{\parskip}{12pt}}

% default link color
\hypersetup{colorlinks, urlcolor={hilit}}

% a few macros
\newcommand{\bi}{\begin{itemize}}
\newcommand{\bbi}{\vspace{24pt} \begin{itemize} \itemsep8pt}
\newcommand{\ei}{\end{itemize}}
\newcommand{\ig}{\includegraphics}
\newcommand{\subt}[1]{{\footnotesize \color{subtitle} {#1}}}
\newcommand{\ttsm}{\tt \small}
\newcommand{\ttfn}{\tt \footnotesize}
\newcommand{\figh}[2]{\centerline{\includegraphics[height=#2\textheight]{#1}}}
\newcommand{\figw}[2]{\centerline{\includegraphics[width=#2\textwidth]{#1}}}


%%%%%%%%%%%%%%%%%%%%%%%%%%%%%%%%%%%%%%%%%%%%%%%%%%%%%%%%%%%%%%%%%%%%%%
% end of header
%%%%%%%%%%%%%%%%%%%%%%%%%%%%%%%%%%%%%%%%%%%%%%%%%%%%%%%%%%%%%%%%%%%%%%

\title{Testing and debugging}
\subtitle{Tools for Reproducible Research}
\author{\href{http://www.biostat.wisc.edu/~kbroman}{Karl Broman}}
\institute{Biostatistics \& Medical Informatics, UW{\textendash}Madison}
\date{\href{http://www.biostat.wisc.edu/~kbroman}{\tt \scriptsize \color{foreground} biostat.wisc.edu/{\textasciitilde}kbroman}
\\[-4pt]
\href{http://github.com/kbroman}{\tt \scriptsize \color{foreground} github.com/kbroman}
\\[-4pt]
\href{https://twitter.com/kwbroman}{\tt \scriptsize \color{foreground} @kwbroman}
\\[-4pt]
{\scriptsize Course web: \href{http://bit.ly/tools4rr}{\tt bit.ly/tools4rr}}
}

\begin{document}

{
\setbeamertemplate{footline}{} % no page number here
\frame{
  \titlepage

\note{We spend a lot of time debugging. We'd spend a lot less time if
  we tested our code properly.

  We want to get the right answers. We can't be sure that we've done
  so without testing our code.

  We want to set up a formal testing system, so that we can be
  confident in our code, and so that problems are identified and
  corrected early.

  Even with a careful testing system, we'll still spend time
  debugging. Debugging can be frustrating, but the right tools and
  skills can speed the process.
}
} }




\begin{frame}[c]{}

\centerline{"I tried it, and it worked."}

\note{This is about the limit of most programmers' testing efforts.

{\nhilit But}: Does it still work? Can you reproduce what you did?
With what variety of inputs did you try it?
}

\end{frame}



\begin{frame}{Types of tests}

\bbi
\onslide<2->{\item Check inputs}
\item \only<2|handout 0>{\lolit} Unit tests
\item \only<2|handout 0>{\lolit} Regression tests
\item \only<2|handout 0>{\lolit} Integration tests
\ei

\note{
}
\end{frame}




\begin{frame}[fragile]{Check inputs}


\begin{lstlisting}
winsorize <-
function(x, q=0.006)
{
  stopifnot(is.numeric(x))
  stopifnot(is.numeric(q), length(q)==1, q>=0, q<=1)

  lohi <- quantile(x, c(q, 1-q), na.rm=TRUE)
  if(diff(lohi) < 0) lohi <- rev(lohi)

  x[!is.na(x) & x < lohi[1]] <- lohi[1]
  x[!is.na(x) & x > lohi[2]] <- lohi[2]
  x
}
\end{lstlisting}


\note{
}
\end{frame}




\begin{frame}[fragile]{assertthat package}


\begin{lstlisting}
#' import assertthat
winsorize <-
function(x, q=0.006)
{
  assert_that(is.numeric(x))
  assert_that(is.number(q), q>=0, q<=1)

  lohi <- quantile(x, c(q, 1-q), na.rm=TRUE)
  if(diff(lohi) < 0) lohi <- rev(lohi)

  x[!is.na(x) & x < lohi[1]] <- lohi[1]
  x[!is.na(x) & x > lohi[2]] <- lohi[2]
  x
}
\end{lstlisting}


\note{
}
\end{frame}




\begin{frame}{Tests in R packages}

\bbi
\item Examples in {\tt .Rd} files
\item Vignettes
\item {\tt tests/} directory
\ei

\note{{\tt R CMD check} is your friend.

(Sure, {\nhilit your} package is correct, but what about all of those
other R packages? If it's not on CRAN, it's {\nhilit probably crap}.)
}
\end{frame}





\begin{frame}[c]{Debugging}

\centerline{Step 1: Reproduce the problem}

\vspace{24pt}

\onslide<2->{\centerline{Step 2: Turn it into a test}}


\note{Try to create the minimal example the produces the problem. This
  helps both for refining your understanding of the problem and for
  speed in testing.

  Once you've created a minimal example that produces the problem,
  {\nhilit add that to your battery of automated tests!} The problem
  may suggest related tests to also add.
}
\end{frame}




\begin{frame}{Learn to use debugging tools}


\bbi
\item {\tt cat}
\item {\tt traceback}
\item RStudio
\item Eclipse
\item gdb
\ei

\note{
}
\end{frame}



\begin{frame}[c]{Debugging}


\centerline{Isolate the problem: where do things go bad?}

\note{``Divide and conquer.''
}
\end{frame}




\begin{frame}[c]{Debugging}


\centerline{Don't make the same mistake twice.}

\note{If you figure out some mistake you've made, search for all other
  possible instances of that mistake.
}
\end{frame}









\begin{frame}[c]{The most pernicious bugs}

\centerline{The code is right, but your thinking is wrong}

\vspace{24pt}

\onslide<2->{\centerline{You don't understand your programming language}}

\vspace{24pt}

\onslide<3->{\centerline{\hilit $\rightarrow$ Write trivial programs to
    test your understanding.}}

\note{Consider that your algorithm may be garbage.
Try an independent implementation.
}
\end{frame}




\begin{frame}[c]{Profiling}

\centerline{Don't try to optimize until the {\hilit very} end}


\note{
}
\end{frame}




\end{document}
