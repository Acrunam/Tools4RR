\documentclass[12pt,t]{beamer}
\usepackage{graphicx}
\setbeameroption{hide notes}
\setbeamertemplate{note page}[plain]

% get rid of junk
\usetheme{default}
\beamertemplatenavigationsymbolsempty
\hypersetup{pdfpagemode=UseNone} % don't show bookmarks on initial view

% font
\usepackage{fontspec}
\setsansfont{TeX Gyre Heros}
\setbeamerfont{note page}{family*=pplx,size=\footnotesize} % Palatino for notes
% "TeX Gyre Heros can be used as a replacement for Helvetica"
% In Unix, unzip the following into ~/.fonts
% In Mac, unzip it, double-click the .otf files, and install using "FontBook"
%   http://www.gust.org.pl/projects/e-foundry/tex-gyre/heros/qhv2.004otf.zip

% named colors
\definecolor{offwhite}{RGB}{249,242,215}
\definecolor{foreground}{RGB}{255,255,255}
\definecolor{background}{RGB}{24,24,24}
\definecolor{title}{RGB}{107,174,214}
\definecolor{gray}{RGB}{155,155,155}
\definecolor{subtitle}{RGB}{102,255,204}
\definecolor{hilight}{RGB}{102,255,204}
\definecolor{vhilight}{RGB}{255,111,207}
\definecolor{lolight}{RGB}{155,155,155}
%\definecolor{green}{RGB}{125,250,125}

% use those colors
\setbeamercolor{titlelike}{fg=title}
\setbeamercolor{subtitle}{fg=subtitle}
\setbeamercolor{institute}{fg=gray}
\setbeamercolor{normal text}{fg=foreground,bg=background}
\setbeamercolor{item}{fg=foreground} % color of bullets
\setbeamercolor{subitem}{fg=gray}
\setbeamercolor{itemize/enumerate subbody}{fg=gray}
\setbeamertemplate{itemize subitem}{{\textendash}}
\setbeamerfont{itemize/enumerate subbody}{size=\footnotesize}
\setbeamerfont{itemize/enumerate subitem}{size=\footnotesize}

% page number
\setbeamertemplate{footline}{%
    \raisebox{5pt}{\makebox[\paperwidth]{\hfill\makebox[20pt]{\color{gray}
          \scriptsize\insertframenumber}}}\hspace*{5pt}}

% add a bit of space at the top of the notes page
\addtobeamertemplate{note page}{\setlength{\parskip}{12pt}}

% a few macros
\newcommand{\bi}{\begin{itemize}}
\newcommand{\ei}{\end{itemize}}
\newcommand{\ig}{\includegraphics}
\newcommand{\subt}[1]{{\footnotesize \color{subtitle} {#1}}}
\newcommand{\ttsm}{\tt \small}

%%%%%%%%%%%%%%%%%%%%%%%%%%%%%%%%%%%%%%%%%%%%%%%%%%%%%%%%%%%%%%%%%%%%%%
% end of header
%%%%%%%%%%%%%%%%%%%%%%%%%%%%%%%%%%%%%%%%%%%%%%%%%%%%%%%%%%%%%%%%%%%%%%

% title info
\title{BMI 826-003}
\subtitle{Tools for Reproducible Research}
\author{\href{http://www.biostat.wisc.edu/~kbroman}{Karl Broman}}
\institute{Biostatistics \& Medical Informatics, UW{\textendash}Madison}
\date{\href{http://www.biostat.wisc.edu/~kbroman}{\tt \scriptsize biostat.wisc.edu/{\textasciitilde}kbroman}
\\[-4pt]
\href{http://github.com/kbroman}{\tt \scriptsize github.com/kbroman}
\\[-4pt]
\href{https://twitter.com/kwbroman}{\tt \scriptsize @kwbroman}
\\[-4pt]
{\scriptsize Course web: \href{http://bit.ly/tools4rr}{\tt \color{hilight} bit.ly/tools4rr}}
}


\begin{document}

% title slide
{
\setbeamertemplate{footline}{} % no page number here
\frame{
  \titlepage
  \note{This is an introductory lecture for a special topics course at
    UW{\textendash}Madison on tools for reproducible research.

A minimal standard for data analysis and other scientific computations
is that they be \emph{reproducible}: that the code and data are assembled
in a way so that another group can re-create all of the results (e.g.,
the figures in a paper). The importance of such reproducibility is now
widely recognized, but it is not so widely practiced as it should be,
in large part because many computational scientists (and particularly
statisticians) have not fully adopted the required tools for
reproducible research.

In this course, we will discuss general principles for reproducible
research but will focus primarily on the use of relevant tools
(particularly \href{http://www.gnu.org/software/make}{\ttsm make},
\href{http://git-scm.com}{\ttsm git}, and \href{http://github.com}{\ttsm knitr}),
with the goal that the students leave the course ready and willing to
ensure that all aspects of their computational research (software,
data analyses, papers, presentations, posters) are reproducible.}
} }


\begin{frame}[c]{}


\centering
\Large 

Reproducible

\bigskip

\onslide<2->{{\color{gray} vs.}}

\bigskip

\only<1|handout 0>{{\color{background} invisible text}}
\only<2>{Replicable}
\only<3 | handout 0>{Correct}

\note{An important distinction: computational work is
  \emph{reproducible\/} if one can take the data and code and produce
  the same set of results. \emph{Replicable\/} is more stringent: can
  someone repeat the experiment and get the same results?

  Reproducibility is a minimal standard. That something is
  reproducible doesn't imply that it is correct. The code may have bugs. The
  methods may be poorly behaved. There could be experimental
  artifacts.

  (But reproducibility is probably correlated with correctness.)
}
\end{frame}



\begin{frame}{Levels of quality}


\vspace{24pt}

\bi
\itemsep12pt
\item Are the tables and figures reproducible from the code and data?
\item Does the code actually do what you think it does?
\item In addition to {\color{hilight} what} was done, is it clear
  {\color{hilight} why} it was done?
  \bi
  \item[] (e.g., how were parameter settings chosen?)
  \ei
\item Can the code be used for other data?
\item Can you extend the code to do other things?
\ei

\note{Reproducibility is not black and white. And the ideal is hard to
  achieve.
}
\end{frame}



\begin{frame}[c]{What is {\color{hilight} raw} data?}


\Large
\centering
How far back to you go?

\note{Data that I get from collaborators has usually gone through a
  considerable amount of pre-processing. Should we have captured that,
  in order for the work to be considered \emph{reproducible}?
}
\end{frame}

\end{document}
