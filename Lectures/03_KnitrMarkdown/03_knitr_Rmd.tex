\documentclass[12pt,t]{beamer}
\usepackage{graphicx}
\setbeameroption{hide notes}
\setbeamertemplate{note page}[plain]
\usepackage{listings}

% set up listing environment
\lstset{language=bash,
        basicstyle=\scriptsize,
        frame=single,
        backgroundcolor=\color{darkgray},
        commentstyle=\color{green},
        showspaces=false,
        showstringspaces=false
        }

% get rid of junk
\usetheme{default}
\beamertemplatenavigationsymbolsempty
\hypersetup{pdfpagemode=UseNone} % don't show bookmarks on initial view


% font
\usepackage{fontspec}
\setsansfont{TeX Gyre Heros}
\setbeamerfont{note page}{family*=pplx,size=\footnotesize} % Palatino for notes
% "TeX Gyre Heros can be used as a replacement for Helvetica"
% In Unix, unzip the following into ~/.fonts
% In Mac, unzip it, double-click the .otf files, and install using "FontBook"
%   http://www.gust.org.pl/projects/e-foundry/tex-gyre/heros/qhv2.004otf.zip

% named colors
\definecolor{offwhite}{RGB}{249,242,215}
\definecolor{foreground}{RGB}{255,255,255}
\definecolor{background}{RGB}{24,24,24}
\definecolor{title}{RGB}{107,174,214}
\definecolor{gray}{RGB}{155,155,155}
\definecolor{subtitle}{RGB}{102,255,204}
\definecolor{hilight}{RGB}{102,255,204}
\definecolor{vhilight}{RGB}{255,111,207}
\definecolor{nhilight}{RGB}{128,0,128}  % hilight color in notes
\definecolor{nvhilight}{RGB}{255,0,128} % vhilight for notes
\definecolor{lolight}{RGB}{155,155,155}
%\definecolor{green}{RGB}{125,250,125}

% use those colors
\setbeamercolor{titlelike}{fg=title}
\setbeamercolor{subtitle}{fg=subtitle}
\setbeamercolor{institute}{fg=gray}
\setbeamercolor{normal text}{fg=foreground,bg=background}
\setbeamercolor{item}{fg=foreground} % color of bullets
\setbeamercolor{subitem}{fg=gray}
\setbeamercolor{itemize/enumerate subbody}{fg=gray}
\setbeamertemplate{itemize subitem}{{\textendash}}
\setbeamerfont{itemize/enumerate subbody}{size=\footnotesize}
\setbeamerfont{itemize/enumerate subitem}{size=\footnotesize}

% page number
\setbeamertemplate{footline}{%
    \raisebox{5pt}{\makebox[\paperwidth]{\hfill\makebox[20pt]{\color{lolight}
          \scriptsize\insertframenumber}}}\hspace*{5pt}}

% add a bit of space at the top of the notes page
\addtobeamertemplate{note page}{\setlength{\parskip}{12pt}}

% default link color
\hypersetup{colorlinks, urlcolor={hilight}}

% a few macros
\newcommand{\bi}{\begin{itemize}}
\newcommand{\ei}{\end{itemize}}
\newcommand{\ig}{\includegraphics}
\newcommand{\subt}[1]{{\footnotesize \color{subtitle} {#1}}}
\newcommand{\ttsm}{\tt \small}

%%%%%%%%%%%%%%%%%%%%%%%%%%%%%%%%%%%%%%%%%%%%%%%%%%%%%%%%%%%%%%%%%%%%%%
% end of header
%%%%%%%%%%%%%%%%%%%%%%%%%%%%%%%%%%%%%%%%%%%%%%%%%%%%%%%%%%%%%%%%%%%%%%

% title info
\title{Writing reproducible reports}
\subtitle{KnitR with R Markdown}
\author{\href{http://www.biostat.wisc.edu/~kbroman}{Karl Broman}}
\institute{Biostatistics \& Medical Informatics, UW{\textendash}Madison}
\date{\href{http://www.biostat.wisc.edu/~kbroman}{\tt \scriptsize \color{white} biostat.wisc.edu/{\textasciitilde}kbroman}
\\[-4pt]
\href{http://github.com/kbroman}{\tt \scriptsize \color{white} github.com/kbroman}
\\[-4pt]
\href{https://twitter.com/kwbroman}{\tt \scriptsize \color{white} @kwbroman}
\\[-4pt]
{\scriptsize Course web: \href{http://bit.ly/tools4rr}{\tt bit.ly/tools4rr}}
}


\begin{document}

% title slide
{
\setbeamertemplate{footline}{} % no page number here
\frame{
  \titlepage

\note{Statisticians write a lot of reports, describing the results of
  data analyses. It's best if such reports are fully reproducible:
  that the data and code are available, and that there's a clear and
  automatic path from data and code to the final report.

  Knitr is ideal for this effort. It's a system for combining code and
  text into a single document. Process the document, and the code is
  replaced with the results and figures that it generates.

  I've found it most efficient to produce informal analysis reports as
  web pages. Markdown is a system for writing simple,
  readable text, with the sort of marks that you might use in an email
  message, that gets converted to nicely formated html-based web pages.

  My goal in this lecture is to show you how to use Knitr with R
  Markdown (a variant of Markdown) to make such
  reproducible reports, and to convince you that this is the way that
  you should be constructing such analysis reports.

  I'd originally planned to also cover KnitR with AsciiDoc, but I
  decided to drop it; it's best to focus on Markdown.
}
} }


\begin{frame}{Data analysis reports}

\vspace{24pt}

\bi
\itemsep24pt
\item Figures/tables + email
\item Static \LaTeX\ or Word document
\item Knitr/Sweave + \LaTeX\ $\rightarrow$ PDF
\item Knitr + Markdown $\rightarrow$ Web page
\ei

\note{Statisticians write a lot of reports. You do a bunch of
  analyses, create a bunch of figures and tables, and you want to
  describe what you've done to a collaborator.

  When I was first starting out, I'd create a bunch of figures and
  tables and email them to my collaborator with a description of the
  findings in the body of the email. That was cumbersome for me and
  for the collaborator. (``Which figure are we talking about, again?'')

  I moved towards writing formal reports in
  \LaTeX\ and sending my collaborator a
  PDF. But that was a lot of work, and if I later wanted to re-run
  things (e.g., if additional data were added), it was a real hassle.

  Sweave + \LaTeX\ was a big help, but it's a pain to deal with page
  breaks.

  Web pages, produced with knitr and Markdown, are ideal. You can make
  super-tall multi-panel figures that show the full details, without
  worrying page breaks. And hyperlinks are more convenient, too.
}
\end{frame}


\end{document}
