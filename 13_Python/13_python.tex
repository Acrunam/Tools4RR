\documentclass[12pt,t]{beamer}
\usepackage{graphicx}
\setbeameroption{hide notes}
\setbeamertemplate{note page}[plain]
\usepackage{listings}

% set up listing environment
\lstset{language=bash,
        basicstyle=\ttfamily\scriptsize,
        frame=single,
        commentstyle=,
        backgroundcolor=\color{darkgray},
        showspaces=false,
        showstringspaces=false
        }

% get rid of junk
\usetheme{default}
\beamertemplatenavigationsymbolsempty
\hypersetup{pdfpagemode=UseNone} % don't show bookmarks on initial view


% font
\usepackage{fontspec}
\setsansfont
  [ ExternalLocation = ../fonts/ ,
    UprightFont = *-regular , 
    BoldFont = *-bold ,
    ItalicFont = *-italic ,
    BoldItalicFont = *-bolditalic ]{texgyreheros}
\setbeamerfont{note page}{family*=pplx,size=\footnotesize} % Palatino for notes
% "TeX Gyre Heros can be used as a replacement for Helvetica"
% I've placed them in ../fonts/; alternatively you can install them
% permanently on your system as follows:
%     Download http://www.gust.org.pl/projects/e-foundry/tex-gyre/heros/qhv2.004otf.zip
%     In Unix, unzip it into ~/.fonts
%     In Mac, unzip it, double-click the .otf files, and install using "FontBook"

% named colors
\definecolor{offwhite}{RGB}{249,242,215}
\definecolor{foreground}{RGB}{255,255,255}
\definecolor{background}{RGB}{24,24,24}
\definecolor{title}{RGB}{107,174,214}
\definecolor{gray}{RGB}{155,155,155}
\definecolor{subtitle}{RGB}{102,255,204}
\definecolor{hilit}{RGB}{102,255,204}
\definecolor{vhilit}{RGB}{255,111,207}
\definecolor{nhilit}{RGB}{128,0,128}  % hilit color in notes
\definecolor{nvhilit}{RGB}{255,0,128} % vhilit for notes
\definecolor{lolit}{RGB}{155,155,155}

\newcommand{\hilit}{\color{hilit}}
\newcommand{\vhilit}{\color{vhilit}}
\newcommand{\nhilit}{\color{nhilit}}
\newcommand{\nvhilit}{\color{nvhilit}}
\newcommand{\lolit}{\color{lolit}}

% use those colors
\setbeamercolor{titlelike}{fg=title}
\setbeamercolor{subtitle}{fg=subtitle}
\setbeamercolor{institute}{fg=gray}
\setbeamercolor{normal text}{fg=foreground,bg=background}
\setbeamercolor{item}{fg=foreground} % color of bullets
\setbeamercolor{subitem}{fg=gray}
\setbeamercolor{itemize/enumerate subbody}{fg=gray}
\setbeamertemplate{itemize subitem}{{\textendash}}
\setbeamerfont{itemize/enumerate subbody}{size=\footnotesize}
\setbeamerfont{itemize/enumerate subitem}{size=\footnotesize}

% page number
\setbeamertemplate{footline}{%
    \raisebox{5pt}{\makebox[\paperwidth]{\hfill\makebox[20pt]{\lolit
          \scriptsize\insertframenumber}}}\hspace*{5pt}}

% add a bit of space at the top of the notes page
\addtobeamertemplate{note page}{\setlength{\parskip}{12pt}}

% default link color
\hypersetup{colorlinks, urlcolor={hilit}}

% a few macros
\newcommand{\bi}{\begin{itemize}}
\newcommand{\bbi}{\vspace{24pt} \begin{itemize} \itemsep8pt}
\newcommand{\ei}{\end{itemize}}
\newcommand{\ig}{\includegraphics}
\newcommand{\subt}[1]{{\footnotesize \color{subtitle} {#1}}}
\newcommand{\ttsm}{\tt \small}
\newcommand{\ttfn}{\tt \footnotesize}
\newcommand{\figh}[2]{\centerline{\includegraphics[height=#2\textheight]{#1}}}
\newcommand{\figw}[2]{\centerline{\includegraphics[width=#2\textwidth]{#1}}}


%%%%%%%%%%%%%%%%%%%%%%%%%%%%%%%%%%%%%%%%%%%%%%%%%%%%%%%%%%%%%%%%%%%%%%
% end of header
%%%%%%%%%%%%%%%%%%%%%%%%%%%%%%%%%%%%%%%%%%%%%%%%%%%%%%%%%%%%%%%%%%%%%%

\title{Python}
\subtitle{Tools for Reproducible Research}
\author{\href{http://kbroman.org}{Karl Broman}}
\institute{Biostatistics \& Medical Informatics, UW{\textendash}Madison}
\date{\href{http://kbroman.org}{\tt \scriptsize \color{foreground} kbroman.org}
\\[-4pt]
\href{http://github.com/kbroman}{\tt \scriptsize \color{foreground} github.com/kbroman}
\\[-4pt]
\href{https://twitter.com/kwbroman}{\tt \scriptsize \color{foreground} @kwbroman}
\\[-4pt]
{\scriptsize Course web: \href{http://kbroman.org/Tools4RR}{\tt kbroman.org/Tools4RR}}
}

\begin{document}

{
\setbeamertemplate{footline}{} % no page number here
\frame{
  \titlepage

\note{
  I'm a big proponent of the use of multiple programming languages:
  use different languages for different types of tasks.

  Statisticians, in particular, should be proficient in some
  ``scripting language'' (e.g., Perl, Python, or Ruby). These types of
  languages give you far more flexibility for manipulating data files.

  I've long used Perl, but I've switched to Ruby, and I'm trying to
  also be proficient in Python. I prefer Ruby to Python, but Python is
  much more widely used, and so if you're going to just learn one such
  language, learn Python.
}
} }


\begin{frame}{Why python?}

\bbi
\item Manipulating data files
\item Simulations using others' programs
\onslide<2->{\item Web-related stuff}
\onslide<3->{
\item Alternative to R for data analysis and graphics
\item Jupyter notebooks
}
\ei

\note{
  For statisticians, the most important use of Python is for the
  manipulation of data files. This sort of script language is great
  for manipulating text, and data files are mostly plain text files.

  In addition, I find a scripting language critical for performing
  simulations to evaluate others' command-line-based programs. They're
  also good for web-related stuff.

  Python can also serve as an alternative to R for data analysis and
  graphics. And Jupyter notebooks are a big deal for reproducible
  research (and they can be used more broadly than Python).
}
\end{frame}


\begin{frame}{Python 2 vs Python 3}

\bbi
\item Most people are using Python version 2.7
\item Python 3 was introduced in {\vhilit 2008}
  \bi
  \item A number of large changes
  \item Some important Python programs haven't been ported
  \item Few people seem to be using it day-to-day
  \ei
\item You should probably stick with Python 2
  \bi
  \item But be aware of differences
  \ei
\ei

\note{
  The biggest annoyance about Python is the two competing versions,
  Python 2 and Python 3. For now, you should probably stick with
  Python 2.

  Python 3 is much better than Python 2, but it hasn't penetrated the
  Python community sufficiently.
}
\end{frame}


\begin{frame}{Installing Python}

\bbi
\item On Mac or Unix, Python should be pre-installed
  \bi
  \item[] {\tt python --version}
  \ei
\item For Windows (or to be current, or to alternate between Python 2
  and 3), install \href{https://store.continuum.io/cshop/anaconda}{Anaconda}
  \bi
  \item[] Includes NumPy, SciPy, Pandas, iPython, Matplotlib, \dots
  \item[] \href{http://continuum.io/downloads}{\tt
    continuum.io/downloads}
  \ei
\ei

\note{
  When you're just starting to learn, you can just stick with the
  pre-installed version of Python, if you are on some flavor of Unix.

  Long term, I recommend Anaconda, which is an easy-to-install Python
  with basically all of the scientific packages you'd want. Installing
  these by hand seems really painful; installing Anaconda is easy.

  Also, with Anaconda, it's easy to switch between Python 2 and
  Python 3.
}
\end{frame}


\begin{frame}{Learning a new language}

\bbi
\item Find a good book
\item Have good example tasks/problems
\item Play around
\item Force yourself to use the new language
\item Develop a script illustrating different language features
\ei

\note{
  It takes time to learn a new programming language. The only way
  you'll learn it is by forcing yourself to use it regularly.
  You need good, realistic problems to tackle. And it might take you
  just 30 minutes with the language you know but all afternoon in the
  new language. But if you don't force yourself, you'll never learn.

  If you go away from it for a week, you'll be quite rusty when you
  come back. I've found it useful to develop a script that illustrates
  the various language features. (``How do I write a loop again? How
  do I define a function?'') Looking through that, you'll pick it all
  up again quickly. It's harder to look back through a book in the
  same way.
}
\end{frame}



\begin{frame}{Into the thick of it}

\vspace{18pt}

{\hilit Learn Python through one example}

\vspace{18pt}

$\qquad$ {\tt markers.txt}

$\qquad$ {\tt families.txt} $\qquad \longrightarrow \qquad$ {\tt data.gen}

$\qquad$ {\tt genotypes.txt}

\note{
  I can't really hope to teach you Python in 50 minutes, but I'll
  try. In this crash course, I'll go through a medium-sized script to
  combine a few data files and convert them into a different form.

  {\tt markers.txt} contains a list of ordered genetic markers. {\tt
    families.txt} contains information about subjects' familial
  relationships. {\tt genotypes.txt} contains subjects' genotypes.

  We're going to convert these data into the form used by the CRI-MAP
  program (an old program for constructing genetic maps).
}
\end{frame}


\begin{frame}[c,fragile]{Input: \tt markers.txt}

\begin{lstlisting}
D20S103
D20S482
D20S851
D20S604
D20S1143
D20S470
D20S477
D20S478
D20S481
D20S159
D20S480
D20S451
D20S171
D20S164
\end{lstlisting}

\note{
  This is the {\tt markers.txt} file. It just has one marker name per
  line.
}
\end{frame}


\begin{frame}[fragile]{Input: \tt families.txt}

\vspace{18pt}

\begin{lstlisting}
Family Individual Father Mother Sex
     1          1      0      0   1
     1          2      0      0   2
     1          3      1      2   1
     1          4      1      2   2
     1          5      1      2   2
     2          1      0      0   1
     2          2      0      0   2
     2          3      1      2   1
     2          4      1      2   1
     3          1      0      0   1
     3          2      0      0   2
     3          3      1      2   2
     3          4      1      2   1
     3          5      1      2   1
     3          6      1      2   2

...

     5          6      1      2   2
     5          7      1      2   1
\end{lstlisting}

\note{
  This is the {\tt families.txt} file; each line is one subject. In
  the {\tt Father} and {\tt Mother} columns, {\tt 0} indicates
  missing: a founding individual in that family.  In the {\tt Sex}
  column, {\tt 2} = female and {\tt 1} = male.
}
\end{frame}
\begin{frame}[c,fragile]{Input: \tt genotypes.txt}

\begin{lstlisting}
Marker   1-1    1-2    1-3    1-4    1-5    2-1    2-2     ...
D20S103         100/98 98/98  98/98  98/98  100/100100/96  ...
D20S1143 176/172180/176176/180       172/180172/176172/172 ...
D20S159  350/358366/354350/354350/354358/366354/350366/354 ...
D20S164         191/207207/207215/191215/207191/207207/215 ...
D20S171  141/135141/137141/141141/137135/137141/139143/135 ...
D20S451  324/308320/316324/316308/320       308/324312/316 ...
D20S470  306/302302/306302/306306/302302/302302/294310/266 ...
D20S477  256/252260/252252/252       256/252256/252        ...
D20S478         267/263263/263263/263263/267255/271263/247 ...
D20S480         304/284       304/284304/284296/296300/300 ...
D20S481  229/237241/237237/237229/237237/237245/245        ...
D20S482  155/159159/167159/159155/167159/167147/155159/155 ...
D20S604  151/147       147/135151/143151/143       147/143 ...
D20S851  132/140148/144132/144132/148132/148       144/140 ...
\end{lstlisting}

\note{
  The {\tt genotypes.txt} file is a bit ugly. Rows are markers and
  individuals are in fixed-width columns, with the genotypes being two
  numeric alleles separated by a slash. Blank fields correspond to
  missing data.
}
\end{frame}



\begin{frame}[c,fragile]{Output: \tt data.gen}

\begin{lstlisting}
5
14
D20S103
D20S482
...
D20S171
D20S164
1
5
1 0 0 1
0 0 155 159 132 140 151 147 176 172 306 302 256 252 0 0 ...
2 0 0 0
100 98 159 167 148 144 0 0 180 176 302 306 260 252 267 ...
3 2 1 1
98 98 159 159 132 144 147 135 176 180 302 306 252 252 ...
4 2 1 0
98 98 155 167 132 148 151 143 0 0 306 302 0 0 263 263 ...
5 2 1 0
98 98 159 167 132 148 151 143 172 180 302 302 256 252 ...
2
4
...
\end{lstlisting}

\note{
  The file we're converting to, {\tt data.gen} in the format used by
  CRI-MAP, is a bit weird: Number of families, number of markers, the
  marker names in order, and then for each family, the family ID, the
  number of subjects in that family, and then the subjects. For each
  subject, there's a line with individual, mom, dad, and sex ({\tt 0}
  = female, {\tt 1} = male), and then a line with genotype data, with
  two numbers for each marker, with {\tt 0}'s for missing values.
}
\end{frame}

\begin{frame}[c,fragile]{The top of the Python script}

\begin{lstlisting}
#!/usr/bin/env python
# Combine the data in "genotypes.txt", "markers.txt" and
# "families.txt" and convert them into a CRI-MAP .gen file
#
# This is the python 2 version

def read_markers (filename):
  "Read an ordered list of marker names from a file."
  with open(filename, 'r') as f:
    lines = f.readlines()
  return [line.strip() for line in lines]

class Person:
  "Person class, to contain the data on a subject."
  def __init__ (self,family, id, dad, mom, sex):
    self.family = family
    self.id = id
    self.dad = dad
    self.mom = mom
    self.sex = "0" if sex == "2" else sex # convert 1/2 -> 1/0
    self.famid = family + '-' + id
    self.gen = {}
\end{lstlisting}

\note{
  The first line ({\tt \#!/usr/bin/env python}) makes it so you can run
  this script from the command line by just typing its name. Using
  {\tt /usr/bin/env} allows that Python might be located in a
  different place on different systems.

  To make the script executable (on unix), type {\tt chmod +x convert2.py}

  In python, comments begin with {\tt \#} (as in R).

  Instead of using braces to delineate blocks of code, Python uses
  indentation. I was initially turned off by this, but I've been
  converted to the idea (mostly from having written a lot of
  CoffeeScript code). You're going to indent anyway; why not have that
  indentation be meaningful?

  You define functions with {\tt def name (param):}

  Unlike R, functions must have a {\tt return} statement if you want to
  return a value.
}
\end{frame}


\begin{frame}[fragile]{The bottom of the Python script}

\vspace{18pt}

\begin{lstlisting}
if __name__ == '__main__':
  # file names
  gfile = "genotypes.txt" # genotype data
  mfile = "markers.txt"   # list of markers, in order
  ffile = "families.txt"  # family information
  ofile = "data.gen"      # output file

  # read the data
  markers = read_markers(mfile)
  people = read_families(ffile)
  read_genotypes(gfile, people)

  # write the data
  write_genfile(ofile, people, markers)
\end{lstlisting}

\note{
  The {\tt convert2.py} script is just a bunch of function definitions
  (and one {\tt class}).

  This bit at the bottom is executed only if the script is
  run from the command line. It does all of the real work: read in the
  data and then write it back out as a {\tt .gen} file.
}
\end{frame}


\begin{frame}{Write functions \& modules not scripts}

\bbi
\item Write a set of reusable functions
\item Your code will be easier to read
\item You may actually reuse the code, this way
\ei

\note{
  With Python (and R), there's a tendency to write a long mess of a
  script. It's better to focus on writing a set of reusable functions.

  With the given example script, you can use {\tt import convert2} to
  load the functions into python. This is similar to {\tt library()}
  in R.
}
\end{frame}

\begin{frame}[fragile]{Try it out}

\vspace{12pt}

\begin{lstlisting}
$ convert2.py
$ diff data.gen data_save.gen
\end{lstlisting}

\vspace{12pt}

\begin{lstlisting}
$ python         # (or ipython)

>>> import convert2

>>> help(convert2)
>>> help(convert2.read_markers)

>>> markers = convert2.read_markers("markers.txt")
>>> markers[0]
>>> len(markers)
>>> markers[-1]
>>> markers[0:2]
>>> markers[0:-1]
>>> markers[5:]
>>> markers[:5]
>>> markers[0:7:2]

>>> quit()
\end{lstlisting}

\note{
  If you type {\tt convert2.py} from the command line, it will run the
  script and create the {\tt data.gen} file, which you'll see (with
  {\tt diff}) matches the target {\tt data\_save.gen} file.

  Or you can type {\tt python} (or {\tt ipython}) at the command line
  and then import the module and run some of the functions by
  hand. You'll need to refer to the functions with the names preceded
  by {\tt convert2.}, or you can use {\tt from convert2 import *} and
  then skip the {\tt convert2.} part.

  The {\tt read.markers} function reads in the ordered list of markers
  as a vector. (In Python, they call it a list.) Vectors in Python are
  indexed starting at 0. You can use the {\tt len} function (like {\tt
  length()} in R) to get the length.

  You can grab slices with {\tt :}, but note that they {\nhilit don't}
  include the last element in the range.

  And negative values are from the end, with {\tt -1} being the
  {\nhilit last} value.

  Also, you can use {\tt start:end:by}. Remember that {\tt end} is
  {\nhilit not} included.
}
\end{frame}




\begin{frame}[fragile]{Read the marker names}

\vspace{18pt}

\begin{lstlisting}
def read_markers (filename):
  "Read an ordered list of marker names from a file."
  with open(filename, 'r') as f:
    lines = f.readlines()
  return [line.strip() for line in lines]
\end{lstlisting}

\note{
  \vspace{-8pt}
  This is the function to read the ordered list of markers.
  It takes a single argument: the name of the file.

  The first line (between the double-quotes) is a description
  that will be shown if you import the module and type
  {\tt help(convert2.read\_markers)}.
  Strings in Python can be defined using single- or double-quotes,
  just like in R.

  The {\tt with} business looks a bit odd, but it ensures that the
  file will be closed if anything goes wrong. I could just as well
  have written {\tt lines=open(filename).readlines()}
  (The {\tt 'r'}, for reading, is the default.)
  After that bit of code, {\tt lines} contains a vector with one
  marker name per line.

  The last line contains a ``list comprehension.'' It's a sort of
  one-line {\tt for} loop, which applies the {\tt strip()}
  function to each element of the {\tt lines} vector (removing any
  end-of-line character).

  {\tt read\_markers} and {\tt open} are ordinary functions, much like
  those in R. {\tt readlines} and {\tt strip} are object-oriented
  ``methods.'' Think of them as functions where the first argument
  precedes the function name.
}
\end{frame}




\begin{frame}[fragile]{\tt class Person}

\vspace{18pt}

\begin{lstlisting}
class Person:
  "Person class, to contain the data on a subject."
  def __init__ (self, family, id, dad, mom, sex):
    self.family = family
    self.id = id
    self.dad = dad
    self.mom = mom
    self.sex = "0" if sex == "2" else sex # convert 1/2 -> 1/0
    self.famid = family + '-' + id
    self.gen = {}
\end{lstlisting}

\vspace{18pt}

Example use:

\vspace{6pt}

\begin{lstlisting}
ind = Person("1", "3", "1", "2", "2")
\end{lstlisting}


\note{
  I first define a class to contain the data for a single
  subject. It contains a function {\tt \_\_init\_\_} for initializing
  an instance of the class (the data object for a single subject).

  Within that function, {\tt self} refers to the newly defined
  instance of the class, and {\tt self.family}, etc., are the way to
  refer to the elements of the class object.

  We create a new {\tt Person} object by calling \\
  {\tt Person(family, id, dad, mom, sex)}
}
\end{frame}



\begin{frame}[fragile]{\tt read\_families}

\vspace{18pt}

\begin{lstlisting}
def read_families (filename):
  "Read family info and return a hash of people."
  with open(filename, 'r') as file:
    file.readline() # header row
    people = {}
    for line in file:
      vals = line.strip().split()
      person = Person(vals[0],vals[1],vals[2],vals[3],vals[4])
      people[person.famid] = person
  return people
\end{lstlisting}

\note{
  This is the function to read the family information.
  I again use {\tt with open() as file:} to open the file. If anything
  goes wrong, the file will be automatically ``closed.''

  I use {\tt readline} to read (but ignore) the header line.

  {\tt people = \{\}} initializes a ``hash.''  This is like an
  unordered vector that is indexed by strings rather than numeric
  indices. (In Python, it's called a ``dictionary.'') I'm going to
  create a hash of {\tt Person} objects, indicated by strings like
  {\tt "1-2"} for individual {\tt 2} in family {\tt 1}.

  I use a {\tt for} loop over lines in the file. For each line, I
  {\tt strip} off any end-of-line character and then {\tt split} it
  at the white space, into a vector.

  I first call {\tt Person} to define the {\tt person} object, as then
  {\tt person.famid} is defined, and I want to use that as the ``hash
  key.''
}
\end{frame}




\begin{frame}[fragile]{\tt read\_genotypes}

\vspace{18pt}

\begin{lstlisting}
def parse_genotype (string):
  "Clean up string -> genotype"
  string = string.replace(' ', '')
  string = "0/0" if string == "" else string
  return string.replace('/', ' ')

def read_genotypes (filename, people):
  "Read genotype data, fill in genotypes within people hash"
  with open(filename, 'r') as file:

    header = file.readline().strip().split()
    header = header[1:] # omit the first field, "Marker"

    for line in file:
      marker = line[:9].replace(' ', '')
      line = line[9:]
      for i in range(len(header)):
        person = header[i]
        start = i*7
        people[person].gen[marker] = \
          parse_genotype(line[start:(start+7)])
\end{lstlisting}

\note{
  The first function here cleans up a genotype string a bit. It strips
  off any white space, substitutes {\tt 0/0} in the case of a blank,
  and replaces the slash with a space. So a string like {\tt "78/125 "}
  will be converted to {\tt "78 125"}.
  The {\tt replace} function for strings is for doing text
  substitutions: it replaces every instance of its first argument with
  its second argument.

  In the {\tt read\_genotypes} function, I grab all but the first
  element of the header line, which match what I'm using as the keys
  for my {\tt people} hash.
  I then go through the rest of the file, one line at a time: I grab
  the marker name (getting rid of any spaces) and then go
  through the rest of the line, 7 characters at a
  time. {\tt range(n)} returns the vector {\tt [0,} {\tt 1,} \dots{\tt
  , n-1]}.

  If you look back at the {\tt Person} class, you'll see that I'd
  initialized {\tt gen} as a hash (with {\tt self.gen = \{\}}). I'm
  filling this in, indexed by marker names.
  The {\tt read\_genotypes} function doesn't return anything, because
  it modifies the input {\tt person} object (as a ``side effect'').

  Note the backslash in the second-to-last line; this allows me to
  split a long line into two. If I left it off, Python would give an
  error.
}
\end{frame}




\begin{frame}[fragile]{Some helper functions}

\vspace{18pt}

\begin{lstlisting}
def get_families (people):
  "Return a vector of distinct families"
  return set([people[key].family for key in people])

def get_family_members (people, family):
  "Return a vector of famids for subjects within a family."
  return [key for key in people \
          if people[key].family == family]

def writeln (file, line, end="\n"):
  "Write a single line to a file."
  file.write(str(line) + end)
\end{lstlisting}

\note{
  Here are a few helper functions that I use in the last {\tt
  write\_genfile} function.

  {\tt get\_families} returns a vector of {\nhilit distinct} family
  IDs. I use a {\nhilit list comprehension} again, which gives a
  vector with all of the family names. Then {\tt set} turns this into
  a ``set'' of distinct values. It acts here sort of like {\tt
  unique()} in R.

  {\tt get\_family\_members} returns a vector of the famid codes for
  the family members in a given family. I'm using a list comprehension
  again, but with an additional {\tt if} qualification.

  {\tt writeln} is just a little wrapper for the {\tt write}
  function, to write a string to a file. Note that in this function,
  the {\tt end} argument has a default value, much like in R
  functions.
}
\end{frame}




\begin{frame}[c,fragile]{\tt write\_genfile}

\begin{lstlisting}
def write_genfile (filename, people, markers):
  "Write genotype data to a file, in CRI-MAP format."
  with open(filename, 'w') as file:
    families = sorted(get_families(people))
    writeln(file, len(families))

    writeln(file, len(markers))
    for marker in markers:
      writeln(file, marker)

    for family in families:
      writeln(file, family)
      members = sorted(get_family_members(people, family), \
                       key=lambda famid: int(people[famid].id))
      writeln(file, len(members))

      for famid in members:
        person = people[famid]
        writeln(file, "%s %s %s %s" % (person.id, \
                      person.mom, person.dad, person.sex))

        for marker in markers:
          writeln(file, person.gen[marker], " ")
        writeln(file, "")
\end{lstlisting}

\note{
  This is the function to write the CRI-MAP file. There are two
  interesting bits here.

  First, {\tt sorted()} returns a sorted version of a vector.
  I use it twice, the second time with {\tt key=lambda famid: int(people[famid].id)} \\
  which is an anonymous function for sorting {\tt Person} objects by
  their individual IDs (numerically).

  The second interesting bit is\\
  {\tt "\%s \%s \%s \%s" \% (person.id, person.mom, person.dad, person.sex)} \\
  which is like {\tt sprintf}, in that I'm formatting a bunch of stuff as a string.
}
\end{frame}



\begin{frame}[fragile]{The bottom of the Python script}

\vspace{18pt}

\begin{lstlisting}
if __name__ == '__main__':
  # file names
  gfile = "genotypes.txt" # genotype data
  mfile = "markers.txt"   # list of markers, in order
  ffile = "families.txt"  # family information
  ofile = "data.gen"      # output file

  # read the data
  markers = read_markers(mfile)
  people = read_families(ffile)
  read_genotypes(gfile, people)

  # write the data
  write_genfile(ofile, people, markers)
\end{lstlisting}

\note{
  We made it through the whole file; here's the bit at the bottom
  again. This bit is run only if you're executing the script from the
  command line.

  I define the file names, read in the data, and then write it back
  out in a different form.
}
\end{frame}



\begin{frame}{Basic types}

\bbi
\item float
  \bi
  \item[] {\tt x = 0.3}
  \ei
\item int
  \bi
  \item[] {\tt m = 5}
  \ei
\item string
  \bi
  \item[] {\tt s = "blah"}
  \ei
\item bool
  \bi
  \item[] {\tt x = True}
  \item[] {\tt y = False}
  \ei
\item None
  \bi
  \item[] {\tt x = None}
  \ei
\item complex
  \bi
  \item[] {\tt x = 5+0j}
  \ei
\ei

\note{
  These are the basic types. Python 2 also distinguishes between {\tt
  int} and {\tt long}, while Python 3 has just {\tt int}.

  {\tt None} is a null object; you can use it as {\tt NA}.
}
\end{frame}

\begin{frame}[fragile]{Converting between types, and such}

\begin{lstlisting}
n = 5
type(n)

s = str(n)
x = float(n)

"%s %s %s" % (n, s, x)
"%d %d %d" % (n, int(s), x)
"%.2f %.2f %.2f" % (n, float(s), x)

dir(s)
dir(x)

s = "blah"
len(s)
s[2:]
s[:-1]
for ch in s:
  print ch
\end{lstlisting}

\note{
  It's important to be able to convert between types, particularly for
  converting between strings and floats.

  The {\tt \%} operator is particularly handy for formatted output, or
  just to convert things into strings. With {\tt \%s}, numbers will
  be converted to strings, but with {\tt \%d} and {\tt \%f}, strings
  will {\nhilit not} automatically be converted to numbers
  {\textendash} you'll get an error.

  The {\tt type} function returns the type of an object. The {\tt dir}
  function returns a list of the methods

  You can use {\tt len} on strings, and you can subset them like a vector
  (aka list), and you can even loop over the characters in a string.
}
\end{frame}


\begin{frame}[fragile]{Multi-element types}

\bbi
\item list
  \bi
  \item[] {\tt x = [1, 2, 3, None, "blah"]}
  \item[] {\tt y = [ [ 1, 2 ], [ 3, 4, 5 ], 6 ] }
  \ei
\item dictionary
  \bi
  \item[] {\tt h = \{'x': 3, 'y': 5, 'name':"Karl"\}}
  \ei
\item tuple
  \bi
  \item[] {\tt x = (1, [2,3])}
  \ei
\item set
  \bi
  \item[] {\tt S = set([5, 3, 5, 1, 2, 1])}
  \ei
\ei

\note{
  Lists are like lists in R: they're ordered vectors whose elements
  can be basically anything.

  Dictionaries are what I call hashes: an un-ordered list indexed by
  strings (called ``keys'').

  Tuples are like lists, but the contents can't be changed. They're
  useful as return values from a function.

  Sets are lists with only unique values.
}
\end{frame}


\begin{frame}[c,fragile]{matrices as lists of lists}

\begin{lstlisting}
x = [ [1, 2, 3, 4], [5, 6, 7, 8], [9, 10, 11, 12] ]

x[1][3]
\end{lstlisting}

\note{
  The simplest way to handle a matrix is as a list of lists.
  They'd typically be stored by rows, as you'd then index the thing as
  {\tt mat[row][col]}.

  Also see numpy (numpy.org), for formal matrices and matrix methods.
}
\end{frame}


\begin{frame}[fragile]{for loops}

\begin{lstlisting}
vec = range(4)
for x in vec:
  print (x+1)**2

import math
for i in xrange(len(vec)):
  print math.log( vec[i] + 1 )


h = {'x':3, 'y':4, 'z':2}
for k in h:
  print k, h[k]

for k in sorted(h.keys()):
  print k, h[k]

for k,v in h.iteritems():
  print k, v

for v in h.itervalues():
  print v
\end{lstlisting}

\note{
  \vspace{-8pt}
  {\tt for} loops over lists successively take each possible value in
  the list. If you want the indices, you need to create a vector of
  indices with {\tt range}. Or use {\tt xrange}, which is avoids
  actually creating the vector, but rather creates the elements when
  they're needed.

  {\tt xrange} is in Python2 only; in Python 3, just use {\tt
  range}. The Python3 {\tt range} is really the Python2 {\tt xrange}.
  So actually in Python3, the first line needs to be {\tt vec = list(range(4))}.

  {\tt for} loops over dictionaries (aka hashes) successively take
  each possible key. You can use {\tt iteriterms()} to iterate over
  key-value pairs or {\tt itervalues()} to iterate over just the
  values. Like {\tt xrange}, these generate the vector of iteracted
  values as needed rather than in advance. There's also a {\tt
  iterkeys()} method, which is what is used as the default for {\tt
  for} loops with dictionaries.

  Note that these loops with dictionaries will be in arbitrary
  order. If you want a particular order, you first need to create a
  sorted vector of keys.

  In Python3, use {\tt .items()} and {\tt .values()} in place of
  {\tt .iteritems()} and {\tt .itervalues()}, respectively.
}
\end{frame}


\begin{frame}[fragile]{list comprehensions}

\begin{lstlisting}
vec = range(10)
[v**2 for v in vec if v > 5]

h = {'x':3, 'y':4, 'zz':2}
[h[k]**2 for k in h]
[h[k]**2 for k in h if len(k) == 1]
[[k, v**3] for k,v in h.iteritems()]
dict( [[k, v**3] for k,v in h.iteritems()] )

x = [k+1 for k in range(6)]
y = [True, False, True, False, False, False]
[x[i] for i in range(len(x)) if y[i]]
\end{lstlisting}

\note{
  List comprehensions are really useful for transformations or
  subsetting.

  The {\tt dict} function will convert a list of key-value pairs into
  a dictionary.

  You can get by without them. But they can provide compact but
  readable code.

  Note, again, that in Python3 you should use {\tt .items()} in place
  of {\tt .iteritems()}.
}
\end{frame}


\begin{frame}[c,fragile]{More with strings}

\begin{lstlisting}
x = "bread and jam"
y = x.split(" ")
z = " ".join(y)

dir(x)
help(x.index)

x.endswith("jam")
x.startswith("bre")
x.count("a")
x.find("and")
x.find("jelly")
x.index("and")
x.index("jelly")

x.replace("jam", "jelly")

x.capitalize()
x.title()
x.upper()
x.upper().lower()
\end{lstlisting}

\note{
  Python has also sorts of methods for doing things with strings.

  Use {\tt dir} to get a list, eg {\tt dir(str)}, {\tt dir("")}, or
  {\tt dir(x)} where {\tt x} is a string.

  Use {\tt help} to get a description of one of the methods, eg
  {\tt help(str.find)}, {\tt help("".find)}, or {\tt help(x.find)}.
}
\end{frame}



\begin{frame}[fragile]{Regular expressions}

\vspace{18pt}

\begin{lstlisting}
import re

x = "Bread and Jam"
re.findall(r'[A-Z]', x)
re.split(r'[A-Z]', x)
re.sub(r'[A-Z]', '', x)

ph = "555-12-3456"
re.findall(r'-', ph)
re.findall(r'\d+', ph)
re.split(r'\D', ph)
re.sub(r'\D', '', ph)
\end{lstlisting}

\note{
  A big reason to use scripting languages is for regular expression
  facilities. But I find regular expressions cumbersome in Python relative
  to Perl or Ruby.

  There is some messiness about backslashes, so it's best with regular
  expressions to use ``raw strings'' by preceding the string with an
  {\tt r}, as so: {\tt r'blah'}.
}
\end{frame}


\begin{frame}[fragile]{Unit tests: Nose}

\vspace{18pt}

\begin{lstlisting}
# This is nosetest_convert2.py
#
# At command line, type "nosetests nosetest_convert2.py"

from nose.tools import assert_equal
from convert2 import *

def test_parse_genotype():
  assert_equal(parse_genotype("       "),   "0 0")
  assert_equal(parse_genotype("100/98 "),   "100 98")
  assert_equal(parse_genotype("90/96  "),   "90 96")
  assert_equal(parse_genotype("90/ 96  "),  "90 96")
  assert_equal(parse_genotype("  3 / 8  "), "3 8")
\end{lstlisting}

\note{
  Unit tests are important for ensuring the correctness of Python
  code, as with any other programming effort.

  Nose is simple tool for making Python unit tests. The above example
  is a minimal use of the tool.

  At the command line, with tests in the file {\tt
  nosetest\_convert2.py}, you'd type {\tt nosetests
  nosetest\_convert2.py}

  The {\tt nose.tools} module contains a bunch of assertion
  functions. Here, I'm just using {\tt assert\_equal}.
}
\end{frame}


\begin{frame}[fragile]{Unit tests: {\tt unittest}}

\vspace{18pt}

\begin{lstlisting}
#!/usr/bin/env python
# Test one of the functions in convert2.py
#
# on the command line, type "test_convert2.py"

import unittest
from convert2 import *

class check_parse_genotype(unittest.TestCase):
  def test_parse_genotype(self):
    self.assertEqual(parse_genotype("       "),   "0 0")
    self.assertEqual(parse_genotype("100/98 "),   "100 98")
    self.assertEqual(parse_genotype("90/96  "),   "90 96")
    self.assertEqual(parse_genotype("90/ 96  "),  "90 96")
    self.assertEqual(parse_genotype("  3 / 8  "), "3 8")

if __name__ == '__main__':
  unittest.main()
\end{lstlisting}

\note{
  Python also has a built-in {\tt unittest} module, but its use
  requires a bit more gunk. I don't totally understand all of this.

  Like Nose, there are a variety of different assertion functions that
  you can use.
}
\end{frame}





\begin{frame}{Summary}

\bbi
\item Learn a scripting language, like Python
  \bi
  \item Not just for manipulating data files, but worth the effort
    just for that.
  \ei
\item Force yourself to use it
\ei

\note{
  Applied statisticians need to be savvy with data file
  manipulation.

  Don't let your scientific collaborators do any copy-paste to move
  data around; any data file manipulation should be done with
  a computer program.

  In the long run, knowing Python, you'll be more self-sufficient and
  versatile.
}
\end{frame}

\end{document}
