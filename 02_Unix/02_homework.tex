\documentclass[12pt]{article}

\usepackage{times}
\usepackage{xcolor}
\usepackage{hyperref}

\hypersetup{pdfpagemode=UseNone} % don't show bookmarks on initial view
\definecolor{hilit}{RGB}{122,0,128}
\hypersetup{colorlinks, urlcolor={hilit}}
\newcommand{\ttsm}{\tt \small}


% revise margins
\setlength{\headheight}{0.0in}
\setlength{\topmargin}{-0.5in}
\setlength{\headsep}{0.0in}
\setlength{\textheight}{10in}
\setlength{\footskip}{0.0in}
\setlength{\oddsidemargin}{0.0in}
\setlength{\evensidemargin}{0.0in}
\setlength{\textwidth}{6.5in}

\setlength{\parskip}{6pt}
\setlength{\parindent}{0pt}

\begin{document}

\thispagestyle{empty}

\textbf{Tools for Reproducible Research} \\
Week 2 Homework

\bigskip

\begin{enumerate}

\item If you weren't able to get the last problem (on GNU Make) from
  \href{http://kbroman.org/Tools4RR/assets/homework/01_homework.pdf}{last week's homework}
  to work, try again this week.

\item Try out {\ttsm curl}, {\ttsm wc}, {\ttsm grep}, and pipes.

  \begin{enumerate}
  \item Use {\ttsm curl} to download the baby names data from the
    Social Security Administration. \\
    {\ttsm curl -O http://www.ssa.gov/oact/babynames/names.zip}

  \item Check that the file looks okay.\\
    {\ttsm unzip -t names.zip}

  \item Unzip the files into a {\ttsm names/} subdirectory.\\
    {\ttsm unzip names.zip -d names}

  \item How many {\ttsm .txt} files are there?

  \item How many total lines are in those {\ttsm .txt} files?

  \item Across all of the files, how many lines contain the name
    ``Pat''?

  \item How many contain the name ``Pat'' as a boy? As a girl?
  \end{enumerate}


\item The following R script performs an exhaustive permutation test and
  then plots the results. It'll take a few minutes to run.

  \href{http://kbroman.org/assets/homework/02_homework.R}{\ttsm kbroman.org/assets/homework/02\_homework.R}

  Try running it from the command line, in the background.

  {\ttsm curl -O http://kbroman.org/assets/homework/02\_homework.R} \\
  {\ttsm R CMD BATCH 02\_homework.R \&}

\end{enumerate}

\end{document}
