\documentclass[12pt,t]{beamer}
\usepackage{graphicx}
\setbeameroption{hide notes}
\setbeamertemplate{note page}[plain]
\usepackage{listings}

% set up listing environment
\lstset{language=bash,
        basicstyle=\ttfamily\scriptsize,
        frame=single,
        commentstyle=,
        backgroundcolor=\color{darkgray},
        showspaces=false,
        showstringspaces=false
        }

% get rid of junk
\usetheme{default}
\beamertemplatenavigationsymbolsempty
\hypersetup{pdfpagemode=UseNone} % don't show bookmarks on initial view


% font
\usepackage{fontspec}
\setsansfont
  [ ExternalLocation = ../fonts/ ,
    UprightFont = *-regular , 
    BoldFont = *-bold ,
    ItalicFont = *-italic ,
    BoldItalicFont = *-bolditalic ]{texgyreheros}
\setbeamerfont{note page}{family*=pplx,size=\footnotesize} % Palatino for notes
% "TeX Gyre Heros can be used as a replacement for Helvetica"
% I've placed them in ../fonts/; alternatively you can install them
% permanently on your system as follows:
%     Download http://www.gust.org.pl/projects/e-foundry/tex-gyre/heros/qhv2.004otf.zip
%     In Unix, unzip it into ~/.fonts
%     In Mac, unzip it, double-click the .otf files, and install using "FontBook"

% named colors
\definecolor{offwhite}{RGB}{249,242,215}
\definecolor{foreground}{RGB}{255,255,255}
\definecolor{background}{RGB}{24,24,24}
\definecolor{title}{RGB}{107,174,214}
\definecolor{gray}{RGB}{155,155,155}
\definecolor{subtitle}{RGB}{102,255,204}
\definecolor{hilit}{RGB}{102,255,204}
\definecolor{vhilit}{RGB}{255,111,207}
\definecolor{nhilit}{RGB}{128,0,128}  % hilit color in notes
\definecolor{nvhilit}{RGB}{255,0,128} % vhilit for notes
\definecolor{lolit}{RGB}{155,155,155}

\newcommand{\hilit}{\color{hilit}}
\newcommand{\vhilit}{\color{vhilit}}
\newcommand{\nhilit}{\color{nhilit}}
\newcommand{\nvhilit}{\color{nvhilit}}
\newcommand{\lolit}{\color{lolit}}

% use those colors
\setbeamercolor{titlelike}{fg=title}
\setbeamercolor{subtitle}{fg=subtitle}
\setbeamercolor{institute}{fg=gray}
\setbeamercolor{normal text}{fg=foreground,bg=background}
\setbeamercolor{item}{fg=foreground} % color of bullets
\setbeamercolor{subitem}{fg=gray}
\setbeamercolor{itemize/enumerate subbody}{fg=gray}
\setbeamertemplate{itemize subitem}{{\textendash}}
\setbeamerfont{itemize/enumerate subbody}{size=\footnotesize}
\setbeamerfont{itemize/enumerate subitem}{size=\footnotesize}

% page number
\setbeamertemplate{footline}{%
    \raisebox{5pt}{\makebox[\paperwidth]{\hfill\makebox[20pt]{\lolit
          \scriptsize\insertframenumber}}}\hspace*{5pt}}

% add a bit of space at the top of the notes page
\addtobeamertemplate{note page}{\setlength{\parskip}{12pt}}

% default link color
\hypersetup{colorlinks, urlcolor={hilit}}

% a few macros
\newcommand{\bi}{\begin{itemize}}
\newcommand{\bbi}{\vspace{24pt} \begin{itemize} \itemsep8pt}
\newcommand{\ei}{\end{itemize}}
\newcommand{\ig}{\includegraphics}
\newcommand{\subt}[1]{{\footnotesize \color{subtitle} {#1}}}
\newcommand{\ttsm}{\tt \small}
\newcommand{\ttfn}{\tt \footnotesize}
\newcommand{\figh}[2]{\centerline{\includegraphics[height=#2\textheight]{#1}}}
\newcommand{\figw}[2]{\centerline{\includegraphics[width=#2\textwidth]{#1}}}


%%%%%%%%%%%%%%%%%%%%%%%%%%%%%%%%%%%%%%%%%%%%%%%%%%%%%%%%%%%%%%%%%%%%%%
% end of header
%%%%%%%%%%%%%%%%%%%%%%%%%%%%%%%%%%%%%%%%%%%%%%%%%%%%%%%%%%%%%%%%%%%%%%

\title{Big jobs/simulations}
\subtitle{Tools for Reproducible Research}
\author{\href{http://www.biostat.wisc.edu/~kbroman}{Karl Broman}}
\institute{Biostatistics \& Medical Informatics, UW{\textendash}Madison}
\date{\href{http://www.biostat.wisc.edu/~kbroman}{\tt \scriptsize \color{foreground} biostat.wisc.edu/{\textasciitilde}kbroman}
\\[-4pt]
\href{http://github.com/kbroman}{\tt \scriptsize \color{foreground} github.com/kbroman}
\\[-4pt]
\href{https://twitter.com/kwbroman}{\tt \scriptsize \color{foreground} @kwbroman}
\\[-4pt]
{\scriptsize Course web: \href{http://bit.ly/tools4rr}{\tt bit.ly/tools4rr}}
}

\begin{document}

{
\setbeamertemplate{footline}{} % no page number here
\frame{
  \titlepage

\note{
  Reproducibility can be considerably harder for computational tasks
  that take more than just a couple of hours, and in some cases, the
  computations for a project may require years of CPU time (split
  across many computers).

  The problems are: (a) it's hard for someone to re-do all of that
  work, and (b) large-scale calculations tend to be organized in a
  system-dependent way, so even if time weren't a factor, it'd be
  harder to transfer the calculations to another system.

  Simulations have some special issues (e.g., saving the seeds for
  random number generators), and they are notoriously irreproducible.

  We at least want to fully document the process: we want capture the
  exact code and workflow, so the results can be reproduced on the
  same system.  And we want that code to be modular and readable, so
  that it {\nhilit could} be restructured for a different system, if
  necessary.

  I must admit that I don't always do this well. Part of this lecture
  is more of a sketch of what I think one should do rather than what I
  actually do.
}
} }


\begin{frame}{But first\dots}

\vspace{18pt}

Suppose I've just written an R function and it seems to work,
and suppose I noticed a simple way to speed it up.

\bigskip

{\hilit What should I do first?}

\only<2->{
\bbi
\item Make it an R package}
\only<3->{\item Write a test or two}
\only<4->{\item Commit it to a git repository}
\only<2->{\ei}

\note{
  My aim here is to reinforce the things we've been covering in the
  course.

  Everything will be a lot easier if you put the code into an R
  package. For the minimal package, all you need are the {\tt
  DESCRIPTION} and {\tt NAMESPACE} files.

  And before you start editing the code, you should write a small
  test. Then you'll have evidence that it currently works, and
  the tests will help show that it's still working after your modifications.

  And before you start editing the code, {\nvhilit {\tt git commit}} what you
  have so far! If it turns out that your new idea doesn't work, will
  you be able to get back to what you had originally?
}
\end{frame}





\begin{frame}{So what's the big deal?}

\bbi
\item You don't want {\tt knitr} running for a year.

\item You don't want to re-run things if you don't have to.
\ei

\note{
  It may not seem like ``big jobs'' are that big of a deal, but in my
  mind this is the only real difficulty in reproducible research.

  Okay, there's also the difficulty that some data can't be generally
  distributed due to confidentiality issues.

  But aside from subjects' confidentiality, the main problem is
  how to manage and capture the really long-running computational
  analyses where even on the same system it can be a gargantuan effort
  to reproduce the results.
}
\end{frame}



\begin{frame}{Unix basics}

\bbi
\item[] {\tt nice +19 R CMD BATCH input.R output.txt \&}

\item[] {\tt fg}
\item[] {\tt ctrl-Z}
\item[] {\tt bg}

\item[] {\tt ps ux}
\item[] {\tt top}

\item[] {\tt kill}
\item[] {\tt kill -9}

\ei


\note{
  Use {\tt R CMD BATCH} to run an R job in the background.

  Use {\tt \&} to put it in the background.

  Use {\tt nice +19} to give it low priority.

  Use {\tt fg} to bring a job back into the foreground.

  Use {\tt ctrl-Z} to suspend a current job; then use {\tt bg} to put it
  in the background.

  Use {\tt ps ux} or {\tt top} to view current jobs.

  Use {\tt kill} or {\tt kill -9} with a process ID ({\tt PID} in the
  output of {\tt ps} and {\tt top}) to kill a job.

  Note: In my experience, Windows sucks at managing multiple
  processes. Windows XP was not bad at this, but Windows XP is dead.
}
\end{frame}



\begin{frame}[c]{Disk thrashing}

\vspace{40pt}

In computer science, thrashing occurs when a computer's virtual memory
subsystem is in a constant state of paging\onslide<2->{, rapidly
exchanging data in memory for data on disk, to the exclusion of most
application-level processing.}

\vspace{40pt}

\hfill {\textendash} \href{http://en.wikipedia.org/wiki/Thrashing_(computer_science)}{Wikipedia}


\note{
  A common problem is having multiple jobs on a machine attempt to use
  more than the available memory on the machine, so then the machine
  starts swapping data from RAM to disk, and all the jobs slow to a crawl.

  If a machine starts disk thrashing, it can be hard to log on and
  kill the jobs.

  The solution: anticipate (and then watch) memory usage.

  Another thing I did: miscalculated the number of files to be
  produced by a job, by a couple of orders of magnitude. It turns out
  that if you go beyond some limit on the number of files in a
  directory, you can totally kill a storage system.
}
\end{frame}


\begin{frame}{Biggish jobs in knitr}

\bbi
\item Manual caching
\item Built-in {\tt cache=TRUE}
\item Split the work and write a {\tt Makefile}
\ei

\note{
  You can put big computations within an R Markdown file, but
  {\nhilit personally} I don't want to wait more than a couple of
  minutes for it to compile. If it's going to take longer than that,
  I'll split things up.

  And if you are going to have some large computations with knitr, you
  won't want to re-run {\nhilit all} of them every time you make even
  the smallest change to the text!

  That's where you want to {\nhilit cache} some computations: save the
  results and just load the results rather than re-run the
  code. Unless the {\nhilit code} changes, in which case you {\nhilit
  do} want to re-run it (and any other code that may depend on the
  results).
}
\end{frame}


\begin{frame}[c,fragile]{Manual caching}

\begin{lstlisting}
```{r a_code_chunk}
file <- "cache/myfile.RData"

if(file.exists(file)) {
  load(file)
} else{

  ....

  save(object1, object2, object3, file=file)
}
```
\end{lstlisting}

\note{
  This is the ``by hand'' approach. If the file doesn't exist, run the
  relevant code and save the needed results to the file. If the file
  does exist, just load the file and skip the code.

  If you want (or need) to re-run the code, you need to delete the
  file {\nhilit manually}.

  One issue: if you want the code to actually be {\nhilit shown}, you
  need to repeat the code: in a chunk that is shown but isn't run, and
  then in this chunk that is run but isn't shown.

  You need to be very careful about dependencies.
}
\end{frame}


\begin{frame}[c,fragile]{Chunk references}

\begin{lstlisting}
```{r not_shown, eval=FALSE}
code_here <- 0
```

```{r a_code_chunk, echo=FALSE}
file <- "cache/myfile.RData"

if(file.exists(file)) {
  load(file)
} else{
<<not_shown>>
  save(code_here, file=file)
}
```
\end{lstlisting}

\note{
  Here's how I'd avoid repeated code: use chunk references.

  The {\tt <<not\_shown>>} is replaced by the code from that chunk with
  that label.
}
\end{frame}



\begin{frame}[c]{A cache gone bad}

\figh{Figs/cache_gone_bad.png}{0.65}

\note{
  This is the sort of thing that can happen with manual caching.

  This is Fig.\ 11.14 from my book, A guide to QTL mapping with R/qtl.

  I saw it immediately upon flipping through my first paper copy of
  the printed book.

  I'd cached some results, but then changed the underlying software
  in a fundamental way and didn't update the cache.
}
\end{frame}


\begin{frame}[fragile]{Knitr's cache system}

\vspace{18pt}

\begin{lstlisting}
```{r chunk_name, cache=TRUE}
load("a_big_file.RData")
med <- apply(object, 2, median, na.rm=TRUE)
```
\end{lstlisting}

\bbi
\item Chunk is re-run if edited.
\item Otherwise, objects from previous run are loaded.
\item Don't cache things with side effects
  \bi
  \item[] e.g., {\tt options()}, {\tt par()}
  \ei
\ei


\note{
  Knitr has a nice built-in system for caching.

  A chunk with {\tt cache=TRUE} will be run once and then all objects
  saved. In future runs, the code won't be run, but rather the cached
  objects will be loaded.

  But some things {\nvhilit shouldn't} be cached. ``Side effects''
  change the state of things; mostly, this is changing global
  variables. If these are placed in a cached chunk, the side effects
  won't be captured when the cache is loaded.
}
\end{frame}




\begin{frame}[fragile]{Cache dependencies}

\vspace{18pt}

Manual dependencies
\bigskip

\begin{lstlisting}
```{r chunkA, cache=TRUE}
Sys.sleep(2)
x <- 5
```

```{r chunkB, cache=TRUE, dependson="chunkA"}
Sys.sleep(2)
y <- x + 1
```

```{r chunkC, cache=TRUE, dependson="chunkB"}
Sys.sleep(2)
z <- y + 1
```
\end{lstlisting}

\note{
  You can indicate dependencies among chunks. Here, if {\tt chunkA} is
  re-run, the other two will be as well.
}
\end{frame}


\begin{frame}[fragile]{Cache dependencies}

\vspace{18pt}

Automatic dependencies
\bigskip

\begin{lstlisting}
```{r setup, include=FALSE}
opts_chunk$set(autodep = TRUE)
dep_auto()
```
\end{lstlisting}


\note{
  There's also an automatic system for determining dependencies among chunks.

  I've not used it.
}
\end{frame}





\begin{frame}{Parallel computing}

\vspace{18pt}

If your computer has multiple processors, use {\tt library(parallel)}
to make use of them.

\bbi
\item {\tt detectCores()}
\item {\tt \hilit RNGkind("L'Eucyer-CMRG")} and {\tt \hilit mclapply} (Unix/Mac)
\item {\tt \hilit makeCluster}, {\tt \hilit clustersetRNGStream}, {\tt
  \hilit clusterApply}, and {\tt \hilit stopCluster} (Windows)
\ei

\note{
  R has a built-in package for performing parallel computations. A
  number of instances of R are invoked, calculations begun, and then
  the results brought back together.

  The code can be a bit ugly, but it's not so bad once you get used to
  it.

  See the links on the resources page,
  {\tt http://kbroman.github.io/Tools4RR/pages/resources.html}

}
\end{frame}


\begin{frame}{Systems for distributed computing}

\bbi
\item \href{http://research.cs.wisc.edu/htcondor}{HTCondor} and the
  UW-Madison \href{http://chtc.cs.wisc.edu/}{CHTC}
\item \href{http://en.wikipedia.org/wiki/Comparison_of_cluster_software}{Other condor-like systems}
\item "By hand"
  \bi
  \item e.g., perl script + template R script
  \ei
\ei

\note{
  For really big jobs, you'll want to distribute the computations
  across multiple computers.

  At UW-Madison, the main place to look is the Center for
  High Throughput Computing (CHTC), and the HTCondor software. This
  provides a way of distributing enormous numbers of jobs across a
  heterogeneous set of computers and collecting the results. The CHTC
  provides great user support.

  There exists other, similar systems for distributing and managing
  jobs across clusters of computers. But at Madison, everyone uses
  HTCondor.

  My own approach is more primitive: I have a Perl script that
  converts a template R script into a bunch of R input files (by
  replacing every instance of ``{\tt SUB}'' in the template with a
  job-specific index). It also creates a script to set those running.

  In either case, you'd write another R script to combine the results
  from the multiple jobs.
}
\end{frame}


\begin{frame}{Simulations}

\bbi
\item Computer simulations require RNG seeds ({\tt .Random.seed} in R).
\item Multiple parallel jobs need different seeds.
\item Don't rely on the current seed, or on having it generated from
  the clock.
\item Use something like {\hilit \tt set.seed(91820205 + i)}
\item An alternative is create a big batch of simulated data sets in
  advance.
\ei

\note{
  RNG = Random number generator

  Simulations split across multiple CPUs each need their own seed.
  In R, the seed is saved as {\tt .Random.seed}; if you start all of
  the simulations from the same directory, they could all get exactly
  the same seed.

  I tend to include a call to {\tt set.seed} at the top of each R
  script, with the seed being some big number plus an index for the
  job.

  You could, alternatively, generate all of the simulated data sets in
  advance. An advantage of this is that it'd be easier to reproduce
  the results later. Just be sure to save (and document) the code you
  used to generate the data.
}
\end{frame}



\begin{frame}[c]{Save everything}

\bi
\item RNG seeds
\item input
\item output
\item version numbers, with {\tt sessionInfo()}
\item raw results
\item script to combine results
\item combined results
\item {\tt ReadMe} describing the point
\ei

\note{
  This stuff (particularly code input \& text output) doesn't take up
  much space. Compartmentalize it and save it.
}
\end{frame}


\begin{frame}{One Makefile to rule them all}

\bbi
\item Separate directory for each batch of big computations.
\item Makefile that controls the combination of the results (and
  everything else).
\item KnitR-based documents for the analysis/use of those results.
\ei

\note{
  This is what I'm thinking, for projects that involve big
  computations: compartmentalize those big computations into chunks,
  each in a separate directory to contain all of the materials and
  results.

  Have one Makefile that handles the combination of those results as
  well as the compilation of any KnitR-based files that describe and
  carry out the further analyses.

  The Makefile won't capture the entire workflow, but it will indicate
  almost all of it, and the big jobs will be compartmentalized as
  subdirectories, and the source of the major results will be
  indicated in the Makefile.
}
\end{frame}


\begin{frame}{Potential problems}

\bbi
\item Forgetting {\tt save()} in your distributed jobs
\item A bug in the {\tt save()} command
\item {\tt make} clobbers some important results
  \bi
  \item Scripts should refuse to overwrite output files
  \ei
\ei

\note{
  These are common mistakes I make.

  I forget the {\tt save} command and so run a ton of computations and
  then get no results.

  Or my {\tt save} command has an error and so I run a ton of
  computations and then it dies on the last line of the script.

  Or I run {\tt make} and it starts re-running some analysis that I
  don't want it to re-run, and it clobbers some important result and
  then I {\nhilit have} to re-run it.
}
\end{frame}





\begin{frame}{Summary}

\bbi
\item Careful organization and modularization.
\item Save everything.
\item Document everything.
\item Learn the basic skills for distributed computing.
\ei

\note{
  It's important to always end with a summary.

  Research with long-running computations are hard to make fully
  reproducible. Modularize the big jobs and document their
  purpose. And document the relationships among things.
}
\end{frame}


\end{document}
