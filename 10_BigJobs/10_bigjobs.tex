\documentclass[12pt,t]{beamer}
\usepackage{graphicx}
\setbeameroption{hide notes}
\setbeamertemplate{note page}[plain]
\usepackage{listings}

% set up listing environment
\lstset{language=bash,
        basicstyle=\ttfamily\scriptsize,
        frame=single,
        commentstyle=,
        backgroundcolor=\color{darkgray},
        showspaces=false,
        showstringspaces=false
        }

% get rid of junk
\usetheme{default}
\beamertemplatenavigationsymbolsempty
\hypersetup{pdfpagemode=UseNone} % don't show bookmarks on initial view


% font
\usepackage{fontspec}
\setsansfont
  [ ExternalLocation = ../fonts/ ,
    UprightFont = *-regular , 
    BoldFont = *-bold ,
    ItalicFont = *-italic ,
    BoldItalicFont = *-bolditalic ]{texgyreheros}
\setbeamerfont{note page}{family*=pplx,size=\footnotesize} % Palatino for notes
% "TeX Gyre Heros can be used as a replacement for Helvetica"
% I've placed them in ../fonts/; alternatively you can install them
% permanently on your system as follows:
%     Download http://www.gust.org.pl/projects/e-foundry/tex-gyre/heros/qhv2.004otf.zip
%     In Unix, unzip it into ~/.fonts
%     In Mac, unzip it, double-click the .otf files, and install using "FontBook"

% named colors
\definecolor{offwhite}{RGB}{249,242,215}
\definecolor{foreground}{RGB}{255,255,255}
\definecolor{background}{RGB}{24,24,24}
\definecolor{title}{RGB}{107,174,214}
\definecolor{gray}{RGB}{155,155,155}
\definecolor{subtitle}{RGB}{102,255,204}
\definecolor{hilit}{RGB}{102,255,204}
\definecolor{vhilit}{RGB}{255,111,207}
\definecolor{nhilit}{RGB}{128,0,128}  % hilit color in notes
\definecolor{nvhilit}{RGB}{255,0,128} % vhilit for notes
\definecolor{lolit}{RGB}{155,155,155}

\newcommand{\hilit}{\color{hilit}}
\newcommand{\vhilit}{\color{vhilit}}
\newcommand{\nhilit}{\color{nhilit}}
\newcommand{\nvhilit}{\color{nvhilit}}
\newcommand{\lolit}{\color{lolit}}

% use those colors
\setbeamercolor{titlelike}{fg=title}
\setbeamercolor{subtitle}{fg=subtitle}
\setbeamercolor{institute}{fg=gray}
\setbeamercolor{normal text}{fg=foreground,bg=background}
\setbeamercolor{item}{fg=foreground} % color of bullets
\setbeamercolor{subitem}{fg=gray}
\setbeamercolor{itemize/enumerate subbody}{fg=gray}
\setbeamertemplate{itemize subitem}{{\textendash}}
\setbeamerfont{itemize/enumerate subbody}{size=\footnotesize}
\setbeamerfont{itemize/enumerate subitem}{size=\footnotesize}

% page number
\setbeamertemplate{footline}{%
    \raisebox{5pt}{\makebox[\paperwidth]{\hfill\makebox[20pt]{\lolit
          \scriptsize\insertframenumber}}}\hspace*{5pt}}

% add a bit of space at the top of the notes page
\addtobeamertemplate{note page}{\setlength{\parskip}{12pt}}

% default link color
\hypersetup{colorlinks, urlcolor={hilit}}

% a few macros
\newcommand{\bi}{\begin{itemize}}
\newcommand{\bbi}{\vspace{24pt} \begin{itemize} \itemsep8pt}
\newcommand{\ei}{\end{itemize}}
\newcommand{\ig}{\includegraphics}
\newcommand{\subt}[1]{{\footnotesize \color{subtitle} {#1}}}
\newcommand{\ttsm}{\tt \small}
\newcommand{\ttfn}{\tt \footnotesize}
\newcommand{\figh}[2]{\centerline{\includegraphics[height=#2\textheight]{#1}}}
\newcommand{\figw}[2]{\centerline{\includegraphics[width=#2\textwidth]{#1}}}


%%%%%%%%%%%%%%%%%%%%%%%%%%%%%%%%%%%%%%%%%%%%%%%%%%%%%%%%%%%%%%%%%%%%%%
% end of header
%%%%%%%%%%%%%%%%%%%%%%%%%%%%%%%%%%%%%%%%%%%%%%%%%%%%%%%%%%%%%%%%%%%%%%

\title{Big jobs/simulations}
\subtitle{Tools for Reproducible Research}
\author{\href{http://www.biostat.wisc.edu/~kbroman}{Karl Broman}}
\institute{Biostatistics \& Medical Informatics, UW{\textendash}Madison}
\date{\href{http://www.biostat.wisc.edu/~kbroman}{\tt \scriptsize \color{foreground} biostat.wisc.edu/{\textasciitilde}kbroman}
\\[-4pt]
\href{http://github.com/kbroman}{\tt \scriptsize \color{foreground} github.com/kbroman}
\\[-4pt]
\href{https://twitter.com/kwbroman}{\tt \scriptsize \color{foreground} @kwbroman}
\\[-4pt]
{\scriptsize Course web: \href{http://bit.ly/tools4rr}{\tt bit.ly/tools4rr}}
}

\begin{document}

{
\setbeamertemplate{footline}{} % no page number here
\frame{
  \titlepage

\note{
  Reproducibility is a bit harder for computational tasks that take
  more than just a couple of hours.

  And I've had papers where the computations required more than a year
  of CPU time (split across many computers).

  The problems are: (a) it's hard for someone to re-do all of that
  work, and (b) large-scale calculations tend to be organized in a
  system-dependent way, so even if time weren't a factor, it'd be that
  much harder to transfer the calculations to another system.

  Simulations have some special issues (e.g., saving the seeds for
  random number generators), and they are notoriously irreproducible.
}
} }


\begin{frame}{But first\dots}

\vspace{18pt}

Suppose I've just written an R function and it seems to work,
and suppose I noticed a simple way to speed it up.

\bigskip

{\hilit But what should I do first?}

\only<2->{
\bbi
\item Make it an R package}
\only<3->{\item Write a test or two}
\only<4->{\item Commit it to a git repository}
\only<2->{\ei}

\note{
  My point here is to reinforce the things we've been covering in the
  course.

  Everything will be a lot easier if you put the code into an R
  package. For the minimal package, all you need are the {\tt
  DESCRIPTION} and {\tt NAMESPACE} files.

  And before you start editing the code, you should write a small
  test. Then you'll have evidence that it currently works, and
  the tests will help show that it's still working after your modifications.

  And before you start editing the code, {\nvhilit {\tt git commit}} what you
  have to far! If it turns out that your new idea doesn't work, will
  you be able to get back to what you had originally?
}
\end{frame}



\begin{frame}{Unix basics}

\bbi
\item[] {\tt nice +19 R CMD BATCH input.R output.txt \&}

\item[] {\tt fg} 
\item[] {\tt ctrl-Z}
\item[] {\tt bg}

\item[] {\tt ps ux}
\item[] {\tt top}

\item[] {\tt kill}
\item[] {\tt kill -9}

\ei


\note{
  Use {\tt R CMD BATCH} to run an R job in the background.

  Use {\tt \&} to put it in the background.

  Use {\tt nice +19} to give it low priority.

  Use {\tt fg} to bring a job back into the foreground.

  Use {\tt ctrl-Z} to suspend a current job; then use {\tt bg} to put it
  in the background.

  Use {\tt ps ux} or {\tt top} to view current jobs.

  Use {\tt kill} or {\tt kill -9} with a process ID ({\tt PID} in the
  output of {\tt ps} and {\tt top}) to kill a job.
}
\end{frame}


\begin{frame}{So what's the big deal?}

\bbi
\item You don't want {\tt knitr} running for a year.

\item You don't want to re-run things if you don't have to.
\ei

\note{
  It may not seem like ``big jobs'' are that big of a deal, but in my
  mind this is the only real difficulty.

}
\end{frame}

\end{document}
