\documentclass[12pt]{article}

\usepackage{times}
\usepackage{xcolor}
\usepackage{hyperref}

\hypersetup{pdfpagemode=UseNone} % don't show bookmarks on initial view
\definecolor{hilit}{RGB}{122,0,128}
\hypersetup{colorlinks, urlcolor={hilit}}
\newcommand{\ttsm}{\tt \small}


% revise margins
\setlength{\headheight}{0.0in}
\setlength{\topmargin}{-0.5in}
\setlength{\headsep}{0.0in}
\setlength{\textheight}{10in}
\setlength{\footskip}{0.0in}
\setlength{\oddsidemargin}{0.0in}
\setlength{\evensidemargin}{0.0in}
\setlength{\textwidth}{6.5in}

\setlength{\parskip}{6pt}
\setlength{\parindent}{0pt}

\begin{document}

\thispagestyle{empty}

\textbf{Tools for Reproducible Research} \\
Homework 4

\bigskip

\begin{enumerate}

\item Create a git repository for the R Markdown document from your
  \href{}{third homework}.

It would go something like this:

\vspace{-12pt}

\begin{verbatim}
mkdir Homework4
cd Homework4
cp ../homework3.Rmd .
git init
git add homework3.Rmd
git commit
\end{verbatim}

\item Create a new repository at GitHub (perhaps private). Push your
  local repository to GitHub.

Locally, you would do something like this:

\vspace{-12pt}

\begin{verbatim}
git remote add origin https://github.com/username/Homework4
git push -u origin master
\end{verbatim}

\end{enumerate}

\end{document}
