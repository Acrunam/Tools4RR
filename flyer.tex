\documentclass[11pt]{article}

\usepackage{times}
\usepackage{amsmath}

% revise margins
\setlength{\headheight}{0.0in}
\setlength{\topmargin}{0.0in}
\setlength{\headsep}{0.0in}
\setlength{\textheight}{8.65in}
\setlength{\footskip}{0.35in}
\setlength{\oddsidemargin}{0.75in}
\setlength{\evensidemargin}{0.75in}
\setlength{\textwidth}{5in}

\setlength{\parskip}{6pt}
\setlength{\parindent}{0pt}

\begin{document}

\thispagestyle{empty}

\textbf{\large \sffamily Tools for Reproducible Research}

\textbf{\sffamily BMI 826-003, Spring, 2014}

\bigskip
\textbf{\sffamily Course summary}

A minimal standard for data analysis and other scientific computations
is that they be \emph{reproducible}: that the code and data are assembled
in a way so that another group can re-create all of the results (e.g.,
the figures in a paper). The importance of such reproducibility is now
widely recognized, but it is not so widely practiced as it should be,
in large part because many computational scientists (and particularly
statisticians) have not fully adopted the required tools for
reproducible research.

In this course, we will discuss general principles for reproducible
research but will focus primarily on the use of relevant tools
(particularly {\tt make}, {\tt git}, and {\tt knitr}),
with the goal that the students leave the course ready and willing to
ensure that all aspects of their computational research (software,
data analyses, papers, presentations, posters) are reproducible.

\bigskip
\textbf{\sffamily Details}

\begin{tabular}{l@{\hspace{5mm}}l}
\textbf{Instructor}: & Karl Broman, {\tt \small bit.ly/kbroman} \\
\textbf{Prerequisite}: & Some knowledge of R \\
\textbf{Lectures}: & Fridays, 11:00--11:50am \\
\textbf{Website}: & {\tt \small bit.ly/tools4rr}\\
\end{tabular}

\bigskip
\textbf{\sffamily Draft schedule}

\renewcommand{\arraystretch}{1.2}
\begin{tabular}{l@{\hspace{5mm}}l}
Jan 24 & Introduction; basic principles; make \\
Jan 31 & Know the command line; know your editor \\
Feb 7  & Knitr with markdown/asciidoc for basic reports \\
Feb 14 & Big jobs/simulations; caching computations \\
Feb 21 & Version control with git \& github/bitbucket \\
Feb 28 & Capturing exploratory analysis \\
Mar 7  & Writing R packages; roxygen2 \\
Mar 14 & Writing clear code \\
Mar 28 & Software testing and debugging \\
Apr 4 &  Organizing data analysis projects \\
Apr 11 & Knitr with latex for papers \\
Apr 18 & Presentations \\
Apr 25 & Posters \\
May 2 &  Ruby/python/perl for data/text manipulation \\
May 9 &  Software/data licenses, copyright, \& human subjects/HIPPA \\
\end{tabular}

\end{document}
