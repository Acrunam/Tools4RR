\documentclass[12pt,t]{beamer}
\usepackage{graphicx}
\setbeameroption{hide notes}
\setbeamertemplate{note page}[plain]
\usepackage{listings}

% set up listing environment
\lstset{language=bash,
        basicstyle=\ttfamily\scriptsize,
        frame=single,
        commentstyle=,
        backgroundcolor=\color{darkgray},
        showspaces=false,
        showstringspaces=false
        }

% get rid of junk
\usetheme{default}
\beamertemplatenavigationsymbolsempty
\hypersetup{pdfpagemode=UseNone} % don't show bookmarks on initial view


% font
\usepackage{fontspec}
\setsansfont
  [ ExternalLocation = ../fonts/ ,
    UprightFont = *-regular , 
    BoldFont = *-bold ,
    ItalicFont = *-italic ,
    BoldItalicFont = *-bolditalic ]{texgyreheros}
\setbeamerfont{note page}{family*=pplx,size=\footnotesize} % Palatino for notes
% "TeX Gyre Heros can be used as a replacement for Helvetica"
% I've placed them in ../fonts/; alternatively you can install them
% permanently on your system as follows:
%     Download http://www.gust.org.pl/projects/e-foundry/tex-gyre/heros/qhv2.004otf.zip
%     In Unix, unzip it into ~/.fonts
%     In Mac, unzip it, double-click the .otf files, and install using "FontBook"

% named colors
\definecolor{offwhite}{RGB}{249,242,215}
\definecolor{foreground}{RGB}{255,255,255}
\definecolor{background}{RGB}{24,24,24}
\definecolor{title}{RGB}{107,174,214}
\definecolor{gray}{RGB}{155,155,155}
\definecolor{subtitle}{RGB}{102,255,204}
\definecolor{hilit}{RGB}{102,255,204}
\definecolor{vhilit}{RGB}{255,111,207}
\definecolor{nhilit}{RGB}{128,0,128}  % hilit color in notes
\definecolor{nvhilit}{RGB}{255,0,128} % vhilit for notes
\definecolor{lolit}{RGB}{155,155,155}

\newcommand{\hilit}{\color{hilit}}
\newcommand{\vhilit}{\color{vhilit}}
\newcommand{\nhilit}{\color{nhilit}}
\newcommand{\nvhilit}{\color{nvhilit}}
\newcommand{\lolit}{\color{lolit}}

% use those colors
\setbeamercolor{titlelike}{fg=title}
\setbeamercolor{subtitle}{fg=subtitle}
\setbeamercolor{institute}{fg=gray}
\setbeamercolor{normal text}{fg=foreground,bg=background}
\setbeamercolor{item}{fg=foreground} % color of bullets
\setbeamercolor{subitem}{fg=gray}
\setbeamercolor{itemize/enumerate subbody}{fg=gray}
\setbeamertemplate{itemize subitem}{{\textendash}}
\setbeamerfont{itemize/enumerate subbody}{size=\footnotesize}
\setbeamerfont{itemize/enumerate subitem}{size=\footnotesize}

% page number
\setbeamertemplate{footline}{%
    \raisebox{5pt}{\makebox[\paperwidth]{\hfill\makebox[20pt]{\lolit
          \scriptsize\insertframenumber}}}\hspace*{5pt}}

% add a bit of space at the top of the notes page
\addtobeamertemplate{note page}{\setlength{\parskip}{12pt}}

% default link color
\hypersetup{colorlinks, urlcolor={hilit}}

% a few macros
\newcommand{\bi}{\begin{itemize}}
\newcommand{\bbi}{\vspace{24pt} \begin{itemize} \itemsep8pt}
\newcommand{\ei}{\end{itemize}}
\newcommand{\ig}{\includegraphics}
\newcommand{\subt}[1]{{\footnotesize \color{subtitle} {#1}}}
\newcommand{\ttsm}{\tt \small}
\newcommand{\ttfn}{\tt \footnotesize}
\newcommand{\figh}[2]{\centerline{\includegraphics[height=#2\textheight]{#1}}}
\newcommand{\figw}[2]{\centerline{\includegraphics[width=#2\textwidth]{#1}}}


%%%%%%%%%%%%%%%%%%%%%%%%%%%%%%%%%%%%%%%%%%%%%%%%%%%%%%%%%%%%%%%%%%%%%%
% end of header
%%%%%%%%%%%%%%%%%%%%%%%%%%%%%%%%%%%%%%%%%%%%%%%%%%%%%%%%%%%%%%%%%%%%%%

\title{Writing R packages}
\subtitle{Tools for Reproducible Research}
\author{\href{http://www.biostat.wisc.edu/~kbroman}{Karl Broman}}
\institute{Biostatistics \& Medical Informatics, UW{\textendash}Madison}
\date{\href{http://www.biostat.wisc.edu/~kbroman}{\tt \scriptsize \color{foreground} biostat.wisc.edu/{\textasciitilde}kbroman}
\\[-4pt]
\href{http://github.com/kbroman}{\tt \scriptsize \color{foreground} github.com/kbroman}
\\[-4pt]
\href{https://twitter.com/kwbroman}{\tt \scriptsize \color{foreground} @kwbroman}
\\[-4pt]
{\scriptsize Course web: \href{http://bit.ly/tools4rr}{\tt bit.ly/tools4rr}}
}

\begin{document}

{
\setbeamertemplate{footline}{} % no page number here
\frame{
  \titlepage

\note{R packages and the Comprehensive R Archive Network (CRAN) are
  important features of R. R packages provide a simple way to distribute R
  code and documentation. And they really are rather simple to create.

  Write an R package to keep track of the misc.\ R functions
  that you write and reuse. Write an R package to distribute the data
  and software that accompany a paper.

  The most painful part of writing an R package is the construction of
  the documentation files, which are in a special {\tt .Rd}
  format. But the Roxygen2 tool makes this rather easy: you write
  comments preceding each R function, in a specially structured way,
  and then use the Roxygen2 tool to create the {\tt .Rd} files for you.
}
} }




\begin{frame}{Why write an R package?}

\vspace{18pt}

\bbi
\item To distribute R code and documentation
\item To keep track of the misc.\ R functions you write
  {\only<2|handout>{\hilit}and reuse}
\item To distribute data and software accompanying a paper.
\ei

\note{R packages can be big and important.

     But that shouldn't scare you off. Assembling a few R functions
     within a package will make it vastly easier for {\nhilit you} to
     use them regularly. You don't even need to distribute the package.

     And really, the R package system is an incredibly important
     feature of R. Packages on CRAN are basically guaranteed to be
     installable, as they are regularly built, installed, and tested
     on multiple systems.
}
\end{frame}


\begin{frame}[c]{}

\figh{Figs/R-exts.png}{0.9}

\note{The Writing R Extensions manual is the key source for the
  specifications of R packages. It's rough going in parts, but if you want to
  get a package on CRAN, you should read it.
}
\end{frame}


\begin{frame}{A simple example: RSkittleBrewer}

\vspace{6pt}

\figh{Figs/RSkittleBrewer.png}{0.8}

\vspace{6pt}

\hfill {\tt \footnotesize \lolit \href{http://alyssafrazee.com/RSkittleBrewer.html}{alyssafrazee.com/RSkittleBrewer.html}}


\note{I was going to write a short example R package, but Alyssa
  Frazee saved me the effort. Here package is a perfect little example
  to illustrate how to write a package.

  It's also a great example of a small but really useful package.
  One small function could be widely useful; you just need to package
  it and tell people about it.
}
\end{frame}


\begin{frame}[fragile]{R package contents}

\vspace{24pt}

\begin{lstlisting}
RSkittleBrewer/

    DESCRIPTION
    NAMESPACE

    R/RSkittleBrewer.R
    R/plotSkittles.R
    R/plotSmarties.R

    man/RSkittleBrewer.Rd
    man/plotSkittles.Rd
    man/plotSmarties.Rd
\end{lstlisting}

\note{
  In the simplest form, an R package is a directory containing: a {\tt DESCRIPTION}
  file (describing the package), a {\tt NAMESPACE} file (indicating
  which functions are available to users), an {\tt R/} directory
  containing R code in {\tt .R} files, and a {\tt man/} directory
  containing documentation in {\tt .Rd} files.
}
\end{frame}

\begin{frame}[fragile]{{\tt DESCRIPTION} file}

\vspace{24pt}

\begin{lstlisting}
Package: RSkittleBrewer
Version: 1.1
Author: Alyssa Frazee
Maintainer: Alyssa Frazee <afrazee@jhsph.edu>
License: MIT + file LICENSE
Title: fun with R colors
Description: for those times you want to make plots with...
URL: https://github.com/alyssafrazee/RSkittleBrewer
\end{lstlisting}

\note{
  The {\tt DESCRIPTION} file is pretty self-explanatory. It just
  contains basic information about the package and its author.

  The simplest way to create this sort of file is to copy and edit one
  from some other package.

  The only part that might be unclear is the {\tt License} field. You
  need to choose a license. We'll talk about this on the very last day
  of the course.

  For now, I'd suggest choosing between the GPL-3
  (the GNU Public License v3) and MIT licenses. GPL-3 has a
  ``pass-through'' provision: software that incorporates GPL-3 code
  must also be licensed as GPL-3. This is a good but restrictive
  thing. The MIT license is the most bare-bones license possible: it
  basically just says ``Do what you want, but don't blame me.''

  An R package with the MIT license needs to also include a {\tt
  LICENSE} file or R will complain; copy and edit the one from the
  RSkittleBrewer package.
}
\end{frame}



\begin{frame}[fragile]{{\tt NAMESPACE} file}

\vspace{24pt}

\begin{lstlisting}
export(RSkittleBrewer)
export(plotSkittles)
export(plotSmarties)
\end{lstlisting}



\note{
  The {\tt NAMESPACE} file is a bit technical: it tells R what
  functions that will be accessible to users.

  The point of this is to keep different packages from stepping on
  each others' toes.
}
\end{frame}




\begin{frame}[fragile]{An {\tt .Rd} file}

\vspace{12pt}

\begin{lstlisting}
\name{RSkittleBrewer}
\alias{RSkittleBrewer}
\title{Candy-based color palettes}
\description{Vectors of colors corresponding to different
             candies.}
\usage{RSkittleBrewer(flavor = c("original", "tropical",
                      "wildberry", "M&M", "smarties"))
}
\arguments{
  \item{flavor}{Character string for candy-based color
  palette.}
}
\value{Vector of character strings representing the chosen
       set of colors.}
\examples{
plotSkittles()
plotSmarties()
}
\keyword{hplot}
\seealso{ \code{\link{plotSkittles}},
          \code{\link{plotSmarties}} }
\end{lstlisting}

\note{
  The R documentation format is very LaTeX-like.

  It describes what the function does, what its
  arguments are, and what output it produces.

  You can further provide examples (which can also serve as tests) and
  links to related functions. The examples need to be {\nhilit fast}
  ($\ll$ 5 sec), because they're run frequently. (CRAN checks every
  package every day on multiple systems.)

  Writing these help files is tedious! That's where Roxygen2 comes in.
}
\end{frame}


\begin{frame}[fragile]{Building, installing, and checking}

\vspace{24pt}

\begin{lstlisting}
R CMD build RSkittleBrewer
R CMD INSTALL RSkittleBrewer_1.1.tar.gz
R CMD check RSkittleBrewer_1.1.tar.gz

R CMD check --as-cran RSkittleBrewer_1.1.tar.gz

R CMD INSTALL --library=~/Rlibs RSkittleBrewer_1.1.tar.gz
# (~/.Renviron file contains R_LIBS=~/Rlibs)

# On windows:
R CMD INSTALL --build RSkittleBrewer_1.1.tar.gz
\end{lstlisting}

\bigskip
\begin{lstlisting}
# also consider (within R):
library(devtools)
build("/path/to/RSkittleBrewer")
build("/path/to/RSkittleBrewer", binary=TRUE)
\end{lstlisting}

\note{
  To install your package, you first need to {\nhilit build} it. {\tt
    R CMD build} creates the {\tt .tar.gz} source file that you'd
  distribute.

  You then use {\tt R CMD INSTALL} to install the package. You may
  want to use {\tt --library=/some/dir} to install to a different
  directory, in which case you need to set {\tt R\_LIBS} in your {\tt
    {\textasciitilde}/.Renviron} file.

  {\tt R CMD check} runs extensive checks of the package, including
  running all of the examples in the help files.
  Pay attention to any errors, warnings, or notes, and revise the
  package to avoid them. This is particularly true if you want to
  submit the package to CRAN.

  Before submitting to CRAN, be sure to run {\tt R CMD check} with the
  lesser-known {\tt --as-cran} flag, which checks further things.

  On Windows, use {\tt R CMD INSTALL --build} to create a {\tt .zip}
  file of the ``compiled'' package. You can also use the {\tt build()}
  function from the devtools package.
}
\end{frame}


\begin{frame}[fragile]{Roxygen2 comments}

\vspace{24pt}

\begin{lstlisting}
#  RSkittleBrewer
#' Candy-based color palettes
#'
#' Vectors of colors corresponding to different candies.
#'
#' @param flavor Character string for candy-based color palette.
#'
#' @export
#' @return Vector of character strings representing the chosen...
#'
#' @examples
#' plotSkittles()
#' plotSmarties()
#'
#' @seealso \code{\link{plotSkittles}},
#'   \code{\link{plotSmarties}}
#' @keywords hplot
RSkittleBrewer <-
...
\end{lstlisting}


\note{
  The {\tt .Rd} files are a pain to create and maintain. Plus, you'll
  end up writing documentation in two places: in the R code, and then
  in the separate {\tt .Rd} files.

  Roxygen2 is a system for writing structured comments in the R code
  that get converted to {\tt .Rd} files. You just maintain the
  documentation in one place: in the R code.
  The Roxygen comments start with {\tt \#'}. The first line is the
  title; the second is the description.

  The individual {\tt {\textbackslash}item}'s within {\tt arguments} become {\tt
    @param}. The {\tt {\textbackslash}value} field becomes {\tt @return}.

  You still need to know a bit about the {\tt .Rd} format; for
  example, the {\tt {\textbackslash}code\{{\textbackslash}link\{blah\}\}}.

  Roxygen2 will across create the {\tt NAMESPACE} file. That's what
  {\tt @export} is for.

  You may also be interested in the Rd2roxygen package, for converting
  a package with hand-written {\tt .Rd} files to the Roxygen2 system.
}
\end{frame}



\begin{frame}[fragile]{Makefile}

\vspace{24pt}

\begin{lstlisting}
# build package documentation
doc:
  R -e 'library(devtools);document(roclets=c("namespace", "rd"))'
\end{lstlisting}

\note{
  Here's the {\tt Makefile} I use to build the {\tt .Rd} files: I use
  the {\tt document()} function within the {\tt devtools} package.
}
\end{frame}


\begin{frame}[fragile]{\tt .Rbuildignore}

\vspace{24pt}

\begin{lstlisting}
Makefile
\end{lstlisting}

\note{You'll want to include a {\tt .Rbuildignore} file in your
  package directory, containing the single line ``{\tt
    Makefile}''. This tells R about files to {\nhilit not} include
 in the {\tt .tar.gz} file it builds. You'd get complaints from {\tt R
   CMD check} otherwise.
}
\end{frame}


\begin{frame}[fragile]{Include a {\tt README} or {\tt README.md} file}

\vspace{6pt}

\begin{lstlisting}
fun with R Colors
=================

If you want high-quality, scientifically-researched color
schemes for your R plots, check out
[RColorBrewer](http://cran.r-project.org/web/packages/RColorBrewer).
If you want your plots to be colored the same way as packs of
Skittles (or M&Ms), then this package (RSkittleBrewer) is the
way to go.

### install
with `devtools`:

```S
devtools::install_github('RSkittleBrewer', 'alyssafrazee')
```

### use
There are only three functions in this package.

Call `RSkittleBrewer` on a flavor to get a vector of R color
names that correspond to that Skittle flavor.
...
\end{lstlisting}

\note{Include a {\tt README} or {\tt README.md} file; R will ignore
  it, but it will show up on GitHub.

  You can use {\tt ReadMe} or {\tt ReadMe.md}, but then you need to
  include it in the {\tt .Rbuildignore} file or R will complain.
}
\end{frame}


\begin{frame}[c]{}


\centerline{That's it!}

\note{That's all you need to make a package.

  There's more, but you should start with just that stuff.
}
\end{frame}


\begin{frame}{Package vignettes}

\bbi
\item Include \emph{vignettes\/} to demonstrate the use of your
  package.
\item It's simplest to use R Markdown.
  \bi
  \item Create a {\tt vignettes/} subdirectory.
  \item Place a {\tt .Rmd} file there.
  \item The name of the file becomes the name of the vignette.
  \ei
\item Include the following at the top of the {\tt .Rmd} file
  \bi
  \item[] {\lolit \ttfn \%{\textbackslash}VignetteEngine\{knitr::knitr\}}
  \item[] {\lolit \ttfn \%{\textbackslash}VignetteIndexEntry\{Intro to RSkittleBrewer\}}
  \ei

\item Load the package in an initial chunk

  \bi
  \item[] {\lolit \ttfn library(RSkittleBrewer)}
  \ei
\ei

\note{In my experience, people tend not to read detailed
  documentation, but they like tutorials. This is what \emph{vignettes\/}
  are for.

  It's easiest to write vignettes in R Markdown (allowed in R 3.0
  onwards). You create a subdirectory {\tt vignettes/} and put the
  usual sort of {\tt .Rmd} files there. You just need to add a couple
  of special lines to the top of the file: to tell R to use knitr to
  build the vignette, and to give a title to the vignette.

  In your {\tt .Rmd} file, you'll need to load the package before you
  start using it.
}
\end{frame}

\begin{frame}{Package vignettes}

\bbi
\item Add the following lines to your {\tt DESCRIPTION} file.
  \bi
  \item[] {\lolit \ttfn VignetteBuilder: knitr}
  \item[] {\lolit \ttfn Suggests: knitr}
  \ei

\item The following lists the vignettes for a package and then opens
  a selected vignette.

  \bi
  \item[] {\lolit \ttfn library(RSkittleBrewer)}
  \item[] {\lolit \ttfn vignette(package="RSkittleBrewer")}
  \item[] {\lolit \ttfn vignette("RSkittleBrewer", "RSkittleBrewer")}
  \ei
\ei

\note{There's one other little detail: you need to mention knitr as
  the vignette builder in your {\tt DESCRIPTION} file.

  If you submit the package to CRAN, links to the vignettes will be listed on
  web page for the package. You might also refer to them in your {\tt
  README} file.

  Within R, you can use the {\tt vignette} function to view the
  available vignettes for a package and to open a particular
  vignette. (With R Markdown vignettes, the compiled html file
  will be opened in your web browser.)
}
\end{frame}

\begin{frame}{Optional stuff}

\bbi
\item {\tt NEWS} file describing changes in each version of the
  package.
\item {\tt inst/CITATION} file describing how to cite your package.
\item {\tt inst/doc/} directory any sort of misc.\ documentation
  (e.g., pre-compiled computationally heavy vignettes)
\item {\tt src/} directory containing C/C++/Fortran code
\item {\tt demo/} directory with demonstrations (like vignettes, but
  to be executed in real-time).
\item {\tt tests/} and/or {\tt inst/tests/} directories containing
  software tests.
\ei


\note{We could go on and on about packages. Here are some of the key
  additional things to consider adding to your package.

  The {\tt inst/} subdirectory is ignored by R but its contents get
  moved back to the root directory for the package when installed. If
  you want your {\tt README} file to be named {\tt ReadMe}, you'd put
  it here.
}
\end{frame}



\begin{frame}{devtools}

\vspace{12pt}

Get to know the \href{http://github.com/hadley/devtools}{devtools} package.

\vspace{12pt}

\bi
\item {\tt dev\_mode()}
\item {\tt load\_all()}
\item {\tt install\_github()}, {\tt install\_bitbucket}, \dots
\item {\tt document()}
\item {\tt build()}
\item {\tt check()}
\item {\tt check\_doc()}
\item {\tt run\_examples()}
\item {\tt test()} {\lolit \, (next time)}
\ei


\note{I've tried to characterize package development as super simple,
  but all coding involves considerable testing, debugging, and
  exploration, and re-building and re-installing R packages can be
  tedious.

  The devtools package reduces a lot of the hassle. Try it out!

  {\tt dev\_mode()} puts you in ``development mode'' so that package
  installs won't affect your regular R package directory. {\tt load\_all()} does
  the equivalent of installing and reloading a package. (Otherwise,
  you might exit R, re-build the package, re-install, invoke R, and
  re-load the package.)

  You can use {\tt install\_github()} to install a package directly from
  GitHub.

  On Mac OSX Mavericks (10.9), gnutar is gone, and devtools gives an
  error. The simplest solution is to make a symbolic link from tar to
  gnutar:

  \qquad  {\tt sudo ln -s /usr/bin/tar /usr/bin/gnutar}
}
\end{frame}


\begin{frame}{Summary}

\bbi
\item R packages really aren't that hard.
\item R packages are really useful.
  \bi
  \item Distributing software and data
  \item Organizing code for a paper
  \item Organizing your misc.\ R functions
  \ei
\item Look at others' packages, and learn from them.
\item Adopt the tools in the
  \href{http://github.com/hadley/devtools}{devtools} package.
\ei


\note{It's surprising to me just how many plain R scripts are sitting
  on people's web pages. Getting them into a package is really not
  that hard and would make the material {\nhilit much} more useful.
}
\end{frame}

\end{document}
