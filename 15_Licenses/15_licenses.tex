\documentclass[12pt,t]{beamer}
\usepackage{graphicx}
\setbeameroption{hide notes}
\setbeamertemplate{note page}[plain]
\usepackage{listings}

% set up listing environment
\lstset{language=bash,
        basicstyle=\ttfamily\scriptsize,
        frame=single,
        commentstyle=,
        backgroundcolor=\color{darkgray},
        showspaces=false,
        showstringspaces=false
        }

% get rid of junk
\usetheme{default}
\beamertemplatenavigationsymbolsempty
\hypersetup{pdfpagemode=UseNone} % don't show bookmarks on initial view


% font
\usepackage{fontspec}
\setsansfont
  [ ExternalLocation = ../fonts/ ,
    UprightFont = *-regular , 
    BoldFont = *-bold ,
    ItalicFont = *-italic ,
    BoldItalicFont = *-bolditalic ]{texgyreheros}
\setbeamerfont{note page}{family*=pplx,size=\footnotesize} % Palatino for notes
% "TeX Gyre Heros can be used as a replacement for Helvetica"
% I've placed them in ../fonts/; alternatively you can install them
% permanently on your system as follows:
%     Download http://www.gust.org.pl/projects/e-foundry/tex-gyre/heros/qhv2.004otf.zip
%     In Unix, unzip it into ~/.fonts
%     In Mac, unzip it, double-click the .otf files, and install using "FontBook"

% named colors
\definecolor{offwhite}{RGB}{249,242,215}
\definecolor{foreground}{RGB}{255,255,255}
\definecolor{background}{RGB}{24,24,24}
\definecolor{title}{RGB}{107,174,214}
\definecolor{gray}{RGB}{155,155,155}
\definecolor{subtitle}{RGB}{102,255,204}
\definecolor{hilit}{RGB}{102,255,204}
\definecolor{vhilit}{RGB}{255,111,207}
\definecolor{nhilit}{RGB}{128,0,128}  % hilit color in notes
\definecolor{nvhilit}{RGB}{255,0,128} % vhilit for notes
\definecolor{lolit}{RGB}{155,155,155}

\newcommand{\hilit}{\color{hilit}}
\newcommand{\vhilit}{\color{vhilit}}
\newcommand{\nhilit}{\color{nhilit}}
\newcommand{\nvhilit}{\color{nvhilit}}
\newcommand{\lolit}{\color{lolit}}

% use those colors
\setbeamercolor{titlelike}{fg=title}
\setbeamercolor{subtitle}{fg=subtitle}
\setbeamercolor{institute}{fg=gray}
\setbeamercolor{normal text}{fg=foreground,bg=background}
\setbeamercolor{item}{fg=foreground} % color of bullets
\setbeamercolor{subitem}{fg=gray}
\setbeamercolor{itemize/enumerate subbody}{fg=gray}
\setbeamertemplate{itemize subitem}{{\textendash}}
\setbeamerfont{itemize/enumerate subbody}{size=\footnotesize}
\setbeamerfont{itemize/enumerate subitem}{size=\footnotesize}

% page number
\setbeamertemplate{footline}{%
    \raisebox{5pt}{\makebox[\paperwidth]{\hfill\makebox[20pt]{\lolit
          \scriptsize\insertframenumber}}}\hspace*{5pt}}

% add a bit of space at the top of the notes page
\addtobeamertemplate{note page}{\setlength{\parskip}{12pt}}

% default link color
\hypersetup{colorlinks, urlcolor={hilit}}

% a few macros
\newcommand{\bi}{\begin{itemize}}
\newcommand{\bbi}{\vspace{24pt} \begin{itemize} \itemsep8pt}
\newcommand{\ei}{\end{itemize}}
\newcommand{\ig}{\includegraphics}
\newcommand{\subt}[1]{{\footnotesize \color{subtitle} {#1}}}
\newcommand{\ttsm}{\tt \small}
\newcommand{\ttfn}{\tt \footnotesize}
\newcommand{\figh}[2]{\centerline{\includegraphics[height=#2\textheight]{#1}}}
\newcommand{\figw}[2]{\centerline{\includegraphics[width=#2\textwidth]{#1}}}


%%%%%%%%%%%%%%%%%%%%%%%%%%%%%%%%%%%%%%%%%%%%%%%%%%%%%%%%%%%%%%%%%%%%%%
% end of header
%%%%%%%%%%%%%%%%%%%%%%%%%%%%%%%%%%%%%%%%%%%%%%%%%%%%%%%%%%%%%%%%%%%%%%

\title{Licenses; human subjects data}
\subtitle{Tools for Reproducible Research}
\author{\href{http://www.biostat.wisc.edu/~kbroman}{Karl Broman}}
\institute{Biostatistics \& Medical Informatics, UW{\textendash}Madison}
\date{\href{http://www.biostat.wisc.edu/~kbroman}{\tt \scriptsize \color{foreground} biostat.wisc.edu/{\textasciitilde}kbroman}
\\[-4pt]
\href{http://github.com/kbroman}{\tt \scriptsize \color{foreground} github.com/kbroman}
\\[-4pt]
\href{https://twitter.com/kwbroman}{\tt \scriptsize \color{foreground} @kwbroman}
\\[-4pt]
{\scriptsize Course web: \href{http://bit.ly/tools4rr}{\tt bit.ly/tools4rr}}
}

\begin{document}

{
\setbeamertemplate{footline}{} % no page number here
\frame{
  \titlepage

\note{
  An often neglected aspect in discussions of reproducible research:
  software and data need to be licensed. If you want your software and
  data to be reused, you need to provide an explicit license that
  explains exactly how the software and data may be reused.

  I'm going to try to explain the issues and give suggestions about
  licenses to consider. But I {\nhilit just barely} understand what
  I'm talking about, and I'm definitely {\nhilit not a lawyer}. I
  don't guarantee that this (or anything else I say) is correct.

  If you will be sharing data on human subjects, or, for that matter,
  just working with data on human subjects, you need to be extra
  careful. I'll try to sketch the basic concerns.
}
} }


\begin{frame}[c]{}

\centerline{{\hilit Note:} I just barely understand this stuff.}

\note{
  Let me repeat: I {\nhilit just barely} understand what I'm talking
  about, and I'm definitely {\nhilit not a lawyer}. I don't guarantee
  that this (or anything else I say) is correct.
}
\end{frame}


\begin{frame}{Intellectual property}

\bbi
\item Manuscripts/journal articles
\item Books
\item Software
\item Data
\item Ideas, inventions
\item Lab/research notebooks
\item Instructional materials
\item Web sites
\ei

\note{
 Intellectual property is property (that someone can own) that is not
 an actual thing but more the idea of the thing. For example, it's not
 the actual physical book, but the text in the book, in any form. It's
 not an actual physical art work, but any depiction of the content of
 that artwork. This can get pretty complicated; it's best to move on.
 
 Most of what academics produce is intellectual property.
}
\end{frame}


\begin{frame}{IP protection}

\bbi
\item Copyright
\item Patents
\item Trademarks, Trade "dress"
\item Trade secrets
\ei

\note{
  Intellectual property can be protected in various ways. I'm going to
  focus on copyright.

  The key thing here is that an {\nhilit idea} or {\nhilit algorithm}
  can't be copyrighted, but those can be protected with a patent.

  So, for example, copyright protection says that you may not be able
  to copy and use someone's exact code, but you can grab all of the
  ideas and re-implement them in your own way.
}
\end{frame}


\begin{frame}{Copyright}

\bbi
\item Copyright is automatic
\item In "works for hire," the employer holds the copyright
\item In academics, it is customary that researchers control copyright
\ei

\note{
  Since 1976, works you create are automatically copyrighted. That
  includes data, software, papers, books, talks, posters, course syllabi,
  and lecture notes.

  In many job situations, your employer will own the copyright on
  works that you create as part of your employment.

  But in academic settings, it is traditional that researchers retain
  copyright on the works they create. But there are exceptions, and
  universities are increasingly interested in generating income from
  the intellectual property of their faculty. 

  Also, students are treated somewhat different from other
  employees. Academic staff may be treated differently from faculty,
  for that matter.

  Ideally, the exact rules are written down somewhere. Find the rules
  and read them, and ask questions about them.
}
\end{frame}


\begin{frame}{Software licenses}

\bbi
\item Critical if you {\hilit want} your code to be reused
\item I choose between the MIT license and the GPL.
\item {\hilit Don't} use Creative Commons licenses for software!
\ei

\note{
  If you don't indicate a license for your software, others {\hilit
  can't} reuse it. You need to be explicit about whether and how your
  software may be reused, by providing a license.

  I choose between the MIT license and the GNU General Public License
  (GPL). The MIT license is as open as possible: do whatever you want,
  just don't sue me. The GPL is ``viral,'' in thatextends to
  derivative works: software that incorporates code under the GPL must
  also be under the GPL.

  The Creative Commons licenses may work for data, but they're more
  for text, music, video, and such things. They {\nhilit should not}
  be used for code, as they are not compatible with other standard
  software licenses, such as the GPL. That means that you wouldn't be
  able to mix in code that was licensed under the GPL.
}
\end{frame}


\begin{frame}{MIT license}


\note{
}
\end{frame}


\begin{frame}{GPL v3}


\note{
}
\end{frame}


\begin{frame}{Creative Commons licenses}

\note{

}
\end{frame}


\begin{frame}{Data licenses}

\note{
}
\end{frame}


\begin{frame}{Human subjects research}

\note{
}
\end{frame}


\begin{frame}{HIPPA}

\note{
}
\end{frame}


\begin{frame}{Internal Review Boards}

\note{
}
\end{frame}


\begin{frame}{Summary}

\note{
}
\end{frame}


\end{document}
