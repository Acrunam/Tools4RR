\documentclass[12pt,t]{beamer}
\usepackage{graphicx}
\setbeameroption{hide notes}
\setbeamertemplate{note page}[plain]
\usepackage{listings}

% set up listing environment
\lstset{language=bash,
        basicstyle=\ttfamily\scriptsize,
        frame=single,
        commentstyle=,
        backgroundcolor=\color{darkgray},
        showspaces=false,
        showstringspaces=false
        }

% get rid of junk
\usetheme{default}
\beamertemplatenavigationsymbolsempty
\hypersetup{pdfpagemode=UseNone} % don't show bookmarks on initial view


% font
\usepackage{fontspec}
\setsansfont
  [ ExternalLocation = ../fonts/ ,
    UprightFont = *-regular , 
    BoldFont = *-bold ,
    ItalicFont = *-italic ,
    BoldItalicFont = *-bolditalic ]{texgyreheros}
\setbeamerfont{note page}{family*=pplx,size=\footnotesize} % Palatino for notes
% "TeX Gyre Heros can be used as a replacement for Helvetica"
% I've placed them in ../fonts/; alternatively you can install them
% permanently on your system as follows:
%     Download http://www.gust.org.pl/projects/e-foundry/tex-gyre/heros/qhv2.004otf.zip
%     In Unix, unzip it into ~/.fonts
%     In Mac, unzip it, double-click the .otf files, and install using "FontBook"

% named colors
\definecolor{offwhite}{RGB}{249,242,215}
\definecolor{foreground}{RGB}{255,255,255}
\definecolor{background}{RGB}{24,24,24}
\definecolor{title}{RGB}{107,174,214}
\definecolor{gray}{RGB}{155,155,155}
\definecolor{subtitle}{RGB}{102,255,204}
\definecolor{hilit}{RGB}{102,255,204}
\definecolor{vhilit}{RGB}{255,111,207}
\definecolor{nhilit}{RGB}{128,0,128}  % hilit color in notes
\definecolor{nvhilit}{RGB}{255,0,128} % vhilit for notes
\definecolor{lolit}{RGB}{155,155,155}

\newcommand{\hilit}{\color{hilit}}
\newcommand{\vhilit}{\color{vhilit}}
\newcommand{\nhilit}{\color{nhilit}}
\newcommand{\nvhilit}{\color{nvhilit}}
\newcommand{\lolit}{\color{lolit}}

% use those colors
\setbeamercolor{titlelike}{fg=title}
\setbeamercolor{subtitle}{fg=subtitle}
\setbeamercolor{institute}{fg=gray}
\setbeamercolor{normal text}{fg=foreground,bg=background}
\setbeamercolor{item}{fg=foreground} % color of bullets
\setbeamercolor{subitem}{fg=gray}
\setbeamercolor{itemize/enumerate subbody}{fg=gray}
\setbeamertemplate{itemize subitem}{{\textendash}}
\setbeamerfont{itemize/enumerate subbody}{size=\footnotesize}
\setbeamerfont{itemize/enumerate subitem}{size=\footnotesize}

% page number
\setbeamertemplate{footline}{%
    \raisebox{5pt}{\makebox[\paperwidth]{\hfill\makebox[20pt]{\lolit
          \scriptsize\insertframenumber}}}\hspace*{5pt}}

% add a bit of space at the top of the notes page
\addtobeamertemplate{note page}{\setlength{\parskip}{12pt}}

% default link color
\hypersetup{colorlinks, urlcolor={hilit}}

% a few macros
\newcommand{\bi}{\begin{itemize}}
\newcommand{\bbi}{\vspace{24pt} \begin{itemize} \itemsep8pt}
\newcommand{\ei}{\end{itemize}}
\newcommand{\ig}{\includegraphics}
\newcommand{\subt}[1]{{\footnotesize \color{subtitle} {#1}}}
\newcommand{\ttsm}{\tt \small}
\newcommand{\ttfn}{\tt \footnotesize}
\newcommand{\figh}[2]{\centerline{\includegraphics[height=#2\textheight]{#1}}}
\newcommand{\figw}[2]{\centerline{\includegraphics[width=#2\textwidth]{#1}}}


%%%%%%%%%%%%%%%%%%%%%%%%%%%%%%%%%%%%%%%%%%%%%%%%%%%%%%%%%%%%%%%%%%%%%%
% end of header
%%%%%%%%%%%%%%%%%%%%%%%%%%%%%%%%%%%%%%%%%%%%%%%%%%%%%%%%%%%%%%%%%%%%%%

\title{Licenses; human subjects data}
\subtitle{Tools for Reproducible Research}
\author{\href{http://kbroman.org}{Karl Broman}}
\institute{Biostatistics \& Medical Informatics, UW{\textendash}Madison}
\date{\href{http://kbroman.org}{\tt \scriptsize \color{foreground} kbroman.org}
\\[-4pt]
\href{http://github.com/kbroman}{\tt \scriptsize \color{foreground} github.com/kbroman}
\\[-4pt]
\href{https://twitter.com/kwbroman}{\tt \scriptsize \color{foreground} @kwbroman}
\\[-4pt]
{\scriptsize Course web: \href{http://kbroman.org/Tools4RR}{\tt kbroman.org/Tools4RR}}
}

\begin{document}

{
\setbeamertemplate{footline}{} % no page number here
\frame{
  \titlepage

\note{
  An often neglected aspect in discussions of reproducible research:
  software and data need to be licensed. If you want your software and
  data to be reused, you need to provide an explicit license that
  explains exactly how the software and data may be reused.

  I'm going to try to explain the issues and give suggestions about
  licenses to consider. But I'm no expert, and I'm definitely {\nhilit
  not a lawyer}. I don't guarantee that this is entirely correct.

  If you will be sharing data on human subjects, or, for that matter,
  just working with data on human subjects, you need to be extra
  careful. I'll try to sketch the basic concerns.
}
} }


{
\setbeamertemplate{footline}{} % no page number here

\begin{frame}<handout:0>{Course summary}

\vspace{12pt}

\bi
\itemsep8pt
\item Make everything you do script-based
  \bi
  \item code + data $\rightarrow$ product
  \ei
\item Use version control (git and GitHub/Bitbucket)
\item Take your time; organize
\item Write clear code; make R packages
\item Write unit tests
\item Capture exploratory data analysis
  \bi
  \item what you did, saw, and thought (and why)
  \ei
\item KnitR + Markdown for reports
\item KnitR + \LaTeX\ for papers, talks, and posters
\item Use licenses to make reusability clear
\ei

\end{frame}
}

\begin{frame}{Intellectual property}

\addtocounter{framenumber}{-1}

\bbi
\item Manuscripts/journal articles
\item Books
\item Software
\item Data sets
\item Ideas, inventions
\item Lab/research notebooks
\item Instructional materials
\item Web sites
\ei

\note{
 Intellectual property is property (ie, someone can own it) that is not
 an actual thing but more the idea of the thing. For example, it's not
 the actual physical book, but the text in the book. It's
 not an actual physical art work, but any depiction of the content of
 that artwork. This can get pretty complicated; it's best to move on.

 Most of what academics produce is intellectual property.
}
\end{frame}


\begin{frame}{IP protection}

\bbi
\item Copyright
\item Patents
\item Trademarks, Trade "dress"
\item Trade secrets
\ei

\note{
  Different kinds of intellectual property are protected in different
  ways. I'm going to focus on copyright.

  An important point to mention here is that an {\nhilit idea},
  {\nhilit fact} or {\nhilit algorithm} can't be copyrighted. Ideas
  and algorithms can be protected with a patent, but {\nhilit facts}
  (including individual data points) can be neither copyrighted nor
  patented.

  So, for example, copyright protection says that you may not be able
  to copy and use someone's exact code, but you can grab all of the
  ideas and re-implement them in your own way. Unless the ideas or
  algorithms are patented, and then you have to get their permission
  to use them, even if it's your own implementation.
}
\end{frame}


\begin{frame}{Copyright}

\bbi
\item Copyright is automatic
\item In "works for hire," the employer holds the copyright
\item In academics, it is customary that researchers control copyright
\onslide<2->{\item At
  \href{https://grad.wisc.edu/acadpolicy/\#responsibleconductofresearch}{UW-Madison}:
    \vspace{6pt}

    \begin{quote} \footnotesize \lolit
      "Except as required by funding agreements or other university
      policies, the university does not claim ownership rights in the
      intellectual property generated during research by its faculty,
      staff, or students."
    \end{quote}
}

\ei

\note{
  \vspace{-8pt}
  Since 1978, works you create are automatically copyrighted. That
  includes data, software, papers, books, talks, posters, course syllabi,
  and lecture notes. Since 1989, you don't need to include a copyright notice.

  In many job situations, your employer will own the copyright on
  works that you create as part of your employment.  But in academic
  settings, it is traditional that researchers retain copyright on the
  works they create. But there are exceptions, and universities are
  increasingly interested in generating income from the intellectual
  property of their faculty.

  Also, students may be treated somewhat differently from other
  employees. Academic staff may be treated differently from faculty,
  for that matter. Ideally, the exact rules are written down
  somewhere. Find the rules and read them, and ask questions about
  them.

  At UW-Madison, faculty, staff, and students own the intellectual
  property of the works they produce except in some special cases,
  such as instructional materials produced with significant university
  resources. And if you're going to patent something, it must be
  through the Wisconsin Alumni Research Foundation (WARF).
}
\end{frame}


\begin{frame}{Exclusive rights under copyright}

\bbi
\item To make copies of the work
\item To distribute/sell copies of the work
\item To create derivative works
\item To perform the work
\item To display the work publicly
\ei

\note{
  Copyright protection gives the author exclusive rights to the work
  and to derivatives of the work.

  Thus, the default is that no one can copy, modify, or redistribute
  your code.
}
\end{frame}




\begin{frame}{Fair use}

\vspace{18pt}

Reproduction for criticism/commentary, teaching, and research

\vspace{12pt}

\bi
\itemsep8pt
\item For non-commercial or nonprofit educational purposes
\item Can't be a substantial portion of the work
\item Can't affect the value/market of the original work
\ei

\note{
  There are important limitations to copyright protection.

  We are allowed to reproduce portions of a work as part of a
  criticism or commentary, in teaching, or for research.

  The rules aren't precise.

  Quoting from a work is okay. Posting the full thing on the web is
  not.
}
\end{frame}




\begin{frame}[c]{}


\centerline{Breaking copyright \quad $\longleftrightarrow$ \quad plagiarism}

\vspace{36pt}

\onslide<2->{\centerline{These are totally different things.}}


\note{
  Just in case it's not clear: the tradition of citing one's sources
  is really totally different from following copyright law.

  Works in the public domain should still be appropriately cited.
}
\end{frame}





\begin{frame}{Software licenses}

\bbi
\item Critical if you {\hilit want} your code to be reused.
\item Also important to protect yourself from lawsuits.
\item I choose between the MIT license and the GPL.
\item {\hilit Don't} use Creative Commons licenses for software!
\ei

\note{
  If you don't indicate a license for your software, others {\hilit
  can't} reuse it. You need to be explicit about whether and how your
  software may be reused, by providing a license.

  I choose between the MIT license and the GNU General Public License
  (GPL). The MIT license is as open as possible: do whatever you want,
  just don't sue me. The GPL is ``viral'' (they say ``copyleft'') in that extends to
  derivative works: software that incorporates code under the GPL must
  also be under the GPL.

  The Creative Commons licenses may work for data, but they're more
  for text, music, video, and such things. They {\nhilit should not}
  be used for code, as they are not compatible with other standard
  software licenses, such as the GPL. That means that you wouldn't be
  able to mix in code that was licensed under the GPL.
}
\end{frame}


\begin{frame}[c]{}

\vspace{48pt}

\centerline{\large Pick a license, any license}

\vspace{72pt}

\hfill
{\textendash} \href{http://blog.codinghorror.com/pick-a-license-any-license/}{Jeff Atwood}

\note{
  I can't emphasize this enough. If you release your software without
  a license, no one can modify it or incorporate it into their own
  software, as it's under copyright protection.

  If you want your software to be reused, pick a license, and make the
  licensing absolutely clear.
}
\end{frame}


\begin{frame}[c,fragile]{MIT license}

\begin{lstlisting}
Copyright (C) <year> <copyright holders>

Permission is hereby granted, free of charge, to any person
obtaining a copy of this software and associated documentation
files (the "Software"), to deal in the Software without
restriction, including without limitation the rights to use,
copy, modify, merge, publish, distribute, sublicense, and/or
sell copies of the Software, and to permit persons to whom the
Software is furnished to do so, subject to the following
conditions:

The above copyright notice and this permission notice shall be
included in all copies or substantial portions of the Software.

THE SOFTWARE IS PROVIDED "AS IS", WITHOUT WARRANTY OF ANY
KIND, EXPRESS OR IMPLIED, INCLUDING BUT NOT LIMITED TO THE
WARRANTIES OF MERCHANTABILITY, FITNESS FOR A PARTICULAR PURPOSE
AND NONINFRINGEMENT. IN NO EVENT SHALL THE AUTHORS OR COPYRIGHT
HOLDERS BE LIABLE FOR ANY CLAIM, DAMAGES OR OTHER LIABILITY,
WHETHER IN AN ACTION OF CONTRACT, TORT OR OTHERWISE, ARISING
FROM, OUT OF OR IN CONNECTION WITH THE SOFTWARE OR THE USE OR
OTHER DEALINGS IN THE SOFTWARE.
\end{lstlisting}


\note{
  The MIT license basically says: do whatever you want with the
  software, but be sure to include this notice, and don't sue me.

  Those are the key things you want: protect yourself from liability,
  and make plain that people can be free to reuse the software.
}
\end{frame}


\begin{frame}{GPL-3}

\bbi
\item Use, modify, distribute, \dots
\item Don't hold the author liable.
\item Distributions must include the source code.
\item Software incorporating the work {\hilit must also be under GPL-3}.
\ei

\note{
  There is an older GPL-2. Use the GPL-3. It was updated to close
  some loopholes.

  Key additions vs MIT license: distributions of the work or
  derivatives must include source code, and derivatives must also be
  licensed under GPL-3.
}
\end{frame}


\begin{frame}[c,fragile]{For GPL-3, include this}

\begin{lstlisting}
<line with the program's name and a brief idea of what it does.>
Copyright (C) <year>  <name of author>

This program is free software: you can redistribute it and/or
modify it under the terms of the GNU General Public License as
published by the Free Software Foundation, either version 3 of
the License, or (at your option) any later version.

This program is distributed in the hope that it will be useful,
but WITHOUT ANY WARRANTY; without even the implied warranty of
MERCHANTABILITY or FITNESS FOR A PARTICULAR PURPOSE.  See the
GNU General Public License for more details.

You should have received a copy of the GNU General Public
License along with this program.  If not, see
<http://www.gnu.org/licenses/>.
\end{lstlisting}

\note{
  To license your software under the GPL, include a notice like this.
}
\end{frame}


\begin{frame}{Creative Commons licenses}

\bbi
\item CC0 {\lolit (Public Domain)}
\item CC BY {\lolit (Attribution)}
\item CC BY-SA {\lolit (Attribution-ShareAlike)}
\item CC BY-ND {\lolit (Attribution-NoDerivs)}
\item CC BY-NC {\lolit (Attribution-NonCommercial)}
\item CC BY-NC-SA {\lolit (Attribution-NonCommercial-ShareAlike)}
\item CC BY-NC-ND {\lolit (Attribution-NonCommercial-NoDerivs)}
\ei

\note{
  The Creative Commons licenses are really useful for things like
  manuscripts, data files, videos, web sites, and such.

  You shouldn't use them for software, as they can't be mixed with the
  GPL, and they make no explicit mention of source or object code.
  In the FAQ at Creative Commons, they explicitly recommend {\nhilit
  against} the use of CC licenses for software.

  BY means people must cite you as the originator.

  SA means that derivative works must be distributed under the same
  license.

  ND means the work must be distributed in its entirety, without any
  changes.

  NC means the work can't be used in a commercial setting.
}
\end{frame}


\begin{frame}{CC licenses: issues to consider}

\bbi
\item BY may be an unnecessary hassle.
\item CC-BY on a paper would allow a company to include it in a book
  \bi
  \item but maybe you don't care
  \ei
\item ND is {\vhilit really} restrictive
 \bi
 \item all or none
 \item no modifications at all
 \ei
\item NC means people in a company can't use it at all
 \bi
 \item might not be usable within a course
 \ei
\ei

\note{
  There are a lot of issues to consider.

  I'd recommend avoiding ND and probably also NC.

  Personally, I'm going with CC0 (by academic tradition, people should
  still cite you) or CC-BY. It means that a company could grab my
  stuff and make money off of it. But I'm fine with that. I'd rather
  see the results of my efforts put to further use.
}
\end{frame}


\begin{frame}{Data copyright}

\bbi
\item Individual data points are generally considered {\hilit facts}
  \bi
  \item Can't be copyrighted
  \ei
\item Compilations of data can be copyrighted
  \bi
  \item Involves some creativity, so an "original work of authorship"
  \ei
\item But someone can just extract and reformat the data
\item Can assign a license to the data files to prevent extraction and
  redistribution
\item See \href{http://bitlaw.com/copyright/database.html}{\tt bitlaw.com/copyright/database.html}
\ei

\note{
  Data are viewed as facts and so they {\nhilit can't} be copyrighted.

  Your data file or database, though, {\nhilit can} be copyrighted,
  if its compilation involves some creativity, and that would
  generally be true for scientific data files.

  So people can't redistribute your data files unless you say it's
  okay. But they may be able to extract and reformat the data and then
  distribute that.

  If you want to prevent extraction and redistribution, the data files
  need a license, which would say the end user is prohibited from
  extracting data for uses other than intended.
}
\end{frame}



\begin{frame}{Keep data open}

\bbi
\item Cite the source; cite the relevant papers
\item Talk to the originator of the data
  \bi
  \item Even if redistribution is legal, don't piss them off.
  \ei
\item For your own data, use
  \href{https://creativecommons.org/publicdomain/zero/1.0/}{CC0}
  (public domain)
\item If you want more control, talk to a lawyer
\ei

\note{
  Statisticians, in particular, should want data to be openly
  available. And so you should cite the source of data and any
  relevant papers, not just because that's the academic tradition, but
  also because we want to reward, as much as possible, people who
  make data accessible.

  Even if it's perfectly legal for you to re-distribute data, you
  should talk to the originator of the data before doing so. You don't
  want them to get annoyed and then stop distributing data in the
  future.

  If you want data to be reused, just put it in the public domain.
  Don't add any extra complexities, like CC-BY.

  If you want to control reuse or redistribution, talk to a lawyer. It
  seems really complicated.
}
\end{frame}


\begin{frame}{Human subjects research}

\bbi
\only<1-2|handout 0>{\item Avoid human subjects research
  \bi
  \item[] {\hilit \only<1|handout 0>{\color{background}} (just kidding!)}
  \ei
}
\only<3->{
\item If there are humans involved, they're human subjects
  \bi
  \item e.g., surveys
  \ei
\item Human subjects research must be reviewed by an
  Institutional Review Board (IRB)
\item Not everything is {\hilit research}
  \bi
  \item e.g., data used solely in a course
  \ei
\item Most things are research
  \bi
  \item If you publish a paper about it, it's research
  \ei
\item Anonymized data may be {\hilit exempt}
  \bi
  \item But the IRB wants to make that determination
  \ei
}
\ei

\note{
  Any research on human subjects must be reviewed by an IRB.

  If you're considering publishing a paper about it, and if humans are
  involved, it's human subjects research. That includes things like
  surveys. So informed consent, and review by IRB, with a clearly
  defined protocol and protection of data.

  You can do a survey {\nhilit within a class} and it may not be
  research, but then you can't publish the results.

  The NIH considers the analysis of anonymized human data to be ``not
  human subjects research,'' and it may be exempt from full IRB
  review, but IRBs generally want to make such determinations
  themselves: you need to fill out some amount of paperwork.
}
\end{frame}


\begin{frame}{HIPAA}

\bbi
\item HIPAA = Health Insurance Portability and Accountability Act of 1996
\item Special rules about medical data with {\hilit any}
  identifying information
  \bi
  \item Private
  \item Secure
  \ei
\item Full zip code may be considered identifying information.
\item Dates of test results are considered identifying information.
\ei

\note{
  HIPAA is really important, but it's also a real pain.

  The key thing is that medical data with {\nhilit any} identifying
  information needs a whole bunch of paperwork if
  transferred/disclosed, and there need to be special security
  measures.

  And the definition of ``identifying information'' is surprisingly
  broad.
}
\end{frame}



\begin{frame}{Summary}

\bbi
\item Pick a license, any license
\item Use MIT or GPL for software
\item Use CC0 for data
\item Cite sources of software and data
\item Talk to the source of data
\item Be careful with human data
  \bi
  \item If you're unsure, ask for help
  \ei
\ei

\note{
  If you don't {\nhilit license} your software, it can't be modified
  or reused.

  Make data open, and be sure to reward those who make data and
  software accessible.

  Be careful with human data, particularly if there's anything
  remotely identifiable.
}
\end{frame}


\end{document}
