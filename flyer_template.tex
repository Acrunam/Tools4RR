\documentclass[11pt]{article}

\usepackage{times}
\usepackage{amsmath}
\usepackage{hyperref}
\hypersetup{pdfpagemode=UseNone} % don't show bookmarks on initial view
\hypersetup{colorlinks, urlcolor={black}}

% revise margins
\setlength{\headheight}{0.0in}
\setlength{\topmargin}{0.0in}
\setlength{\headsep}{0.0in}
\setlength{\textheight}{8.65in}
\setlength{\footskip}{0.35in}
\setlength{\oddsidemargin}{0.75in}
\setlength{\evensidemargin}{0.75in}
\setlength{\textwidth}{5in}

\setlength{\parskip}{6pt}
\setlength{\parindent}{0pt}

\newcommand{\ttsm}{\tt \small}

\begin{document}

\thispagestyle{empty}

\textbf{\large \sffamily Tools for Reproducible Research}

\textbf{\sffamily BMI 826-003, Spring, 2014}

\bigskip
\textbf{\sffamily Course summary}

A minimal standard for data analysis and other scientific computations
is that they be \emph{reproducible}: that the code and data are assembled
in a way so that another group can re-create all of the results (e.g.,
the figures in a paper). The importance of such reproducibility is now
widely recognized, but it is not so widely practiced as it should be,
in large part because many computational scientists (and particularly
statisticians) have not fully adopted the required tools for
reproducible research.

In this course, we will discuss general principles for reproducible
research but will focus primarily on the use of relevant tools
(particularly \href{http://www.gnu.org/software/make}{\ttsm make},
\href{http://git-scm.com}{\ttsm git}, and \href{http://github.com}{\ttsm knitr}),
with the goal that the students leave the course ready and willing to
ensure that all aspects of their computational research (software,
data analyses, papers, presentations, posters) are reproducible.

\bigskip
\textbf{\sffamily Details}

\begin{tabular}{l@{\hspace{5mm}}l}
\textbf{Instructor}: & Karl Broman, \href{http://www.biostat.wisc.edu/~kbroman}{\ttsm bit.ly/kbroman} \\
\textbf{Prerequisite}: & Some knowledge of R \\
\textbf{Lectures}: & Fridays, 11:00--11:50am, 1116
\href{http://map.wisc.edu/s/psk50tw2}{Biochem}, 420 Henry Mall \\
\textbf{Website}: & \href{http://kbroman.github.io/Tools4RR}{\ttsm bit.ly/tools4rr} \\
\end{tabular}

\bigskip
\textbf{\sffamily Draft schedule}

\renewcommand{\arraystretch}{1.2}
\begin{tabular}{l@{\hspace{5mm}}l}
<<<PUT_SCHEDULE_HERE>>>
\end{tabular}

\end{document}
