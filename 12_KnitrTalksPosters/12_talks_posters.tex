\documentclass[12pt,t]{beamer}
\usepackage{graphicx}
\setbeameroption{hide notes}
\setbeamertemplate{note page}[plain]
\usepackage{listings}

% set up listing environment
\lstset{language=bash,
        basicstyle=\ttfamily\scriptsize,
        frame=single,
        commentstyle=,
        backgroundcolor=\color{darkgray},
        showspaces=false,
        showstringspaces=false
        }

% get rid of junk
\usetheme{default}
\beamertemplatenavigationsymbolsempty
\hypersetup{pdfpagemode=UseNone} % don't show bookmarks on initial view


% font
\usepackage{fontspec}
\setsansfont
  [ ExternalLocation = ../fonts/ ,
    UprightFont = *-regular , 
    BoldFont = *-bold ,
    ItalicFont = *-italic ,
    BoldItalicFont = *-bolditalic ]{texgyreheros}
\setbeamerfont{note page}{family*=pplx,size=\footnotesize} % Palatino for notes
% "TeX Gyre Heros can be used as a replacement for Helvetica"
% I've placed them in ../fonts/; alternatively you can install them
% permanently on your system as follows:
%     Download http://www.gust.org.pl/projects/e-foundry/tex-gyre/heros/qhv2.004otf.zip
%     In Unix, unzip it into ~/.fonts
%     In Mac, unzip it, double-click the .otf files, and install using "FontBook"

% named colors
\definecolor{offwhite}{RGB}{249,242,215}
\definecolor{foreground}{RGB}{255,255,255}
\definecolor{background}{RGB}{24,24,24}
\definecolor{title}{RGB}{107,174,214}
\definecolor{gray}{RGB}{155,155,155}
\definecolor{subtitle}{RGB}{102,255,204}
\definecolor{hilit}{RGB}{102,255,204}
\definecolor{vhilit}{RGB}{255,111,207}
\definecolor{nhilit}{RGB}{128,0,128}  % hilit color in notes
\definecolor{nvhilit}{RGB}{255,0,128} % vhilit for notes
\definecolor{lolit}{RGB}{155,155,155}

\newcommand{\hilit}{\color{hilit}}
\newcommand{\vhilit}{\color{vhilit}}
\newcommand{\nhilit}{\color{nhilit}}
\newcommand{\nvhilit}{\color{nvhilit}}
\newcommand{\lolit}{\color{lolit}}

% use those colors
\setbeamercolor{titlelike}{fg=title}
\setbeamercolor{subtitle}{fg=subtitle}
\setbeamercolor{institute}{fg=gray}
\setbeamercolor{normal text}{fg=foreground,bg=background}
\setbeamercolor{item}{fg=foreground} % color of bullets
\setbeamercolor{subitem}{fg=gray}
\setbeamercolor{itemize/enumerate subbody}{fg=gray}
\setbeamertemplate{itemize subitem}{{\textendash}}
\setbeamerfont{itemize/enumerate subbody}{size=\footnotesize}
\setbeamerfont{itemize/enumerate subitem}{size=\footnotesize}

% page number
\setbeamertemplate{footline}{%
    \raisebox{5pt}{\makebox[\paperwidth]{\hfill\makebox[20pt]{\lolit
          \scriptsize\insertframenumber}}}\hspace*{5pt}}

% add a bit of space at the top of the notes page
\addtobeamertemplate{note page}{\setlength{\parskip}{12pt}}

% default link color
\hypersetup{colorlinks, urlcolor={hilit}}

% a few macros
\newcommand{\bi}{\begin{itemize}}
\newcommand{\bbi}{\vspace{24pt} \begin{itemize} \itemsep8pt}
\newcommand{\ei}{\end{itemize}}
\newcommand{\ig}{\includegraphics}
\newcommand{\subt}[1]{{\footnotesize \color{subtitle} {#1}}}
\newcommand{\ttsm}{\tt \small}
\newcommand{\ttfn}{\tt \footnotesize}
\newcommand{\figh}[2]{\centerline{\includegraphics[height=#2\textheight]{#1}}}
\newcommand{\figw}[2]{\centerline{\includegraphics[width=#2\textwidth]{#1}}}


%%%%%%%%%%%%%%%%%%%%%%%%%%%%%%%%%%%%%%%%%%%%%%%%%%%%%%%%%%%%%%%%%%%%%%
% end of header
%%%%%%%%%%%%%%%%%%%%%%%%%%%%%%%%%%%%%%%%%%%%%%%%%%%%%%%%%%%%%%%%%%%%%%

\title{Presentations and posters}
\subtitle{Tools for Reproducible Research}
\author{\href{http://www.biostat.wisc.edu/~kbroman}{Karl Broman}}
\institute{Biostatistics \& Medical Informatics, UW{\textendash}Madison}
\date{\href{http://www.biostat.wisc.edu/~kbroman}{\tt \scriptsize \color{foreground} biostat.wisc.edu/{\textasciitilde}kbroman}
\\[-4pt]
\href{http://github.com/kbroman}{\tt \scriptsize \color{foreground} github.com/kbroman}
\\[-4pt]
\href{https://twitter.com/kwbroman}{\tt \scriptsize \color{foreground} @kwbroman}
\\[-4pt]
{\scriptsize Course web: \href{http://bit.ly/tools4rr}{\tt bit.ly/tools4rr}}
}

\begin{document}

{
\setbeamertemplate{footline}{} % no page number here
\frame{
  \titlepage

\note{
  It's arguably less critical that presentation slides or a poster be
  reproducible. Nevertheless, there can be great personal advantage to
  the automated generation of figures and such in slides or a poster:
  if the primary data should change, or if some analysis mistake is
  discovered, it will be easier to revise the presentation.

  My primary goal is to get you to ditch Powerpoint/Keynote in favor
  of reproducible alternatives. I will primarily focus on the Beamer
  package for LaTeX, for both slides and posters. But I will also
  touch upon the use of slidify to make Markdown-based slides for a
  talk.
}
} }


\begin{frame}{Powerpoint/Keynote}

\begin{minipage}[t]{0.45\textwidth}

\bbi
\item[+] Standard
\item[+] Easy to share slides
\item[+] WYSIWYG (mostly)
\item[+] Fancy animations
\ei

\end{minipage}
\hfill
\begin{minipage}[t]{0.45\textwidth}

\bbi
\item[\textendash] Font problems
\item[\textendash] Lots of copy-paste
\item[\textendash] Hard to get equations
\item[\textendash] Not reproducible
\ei

\end{minipage}

\note{
  Powerpoint and Keynote do have their advantages, the principal one
  being that everyone is using these tools, which makes it easy to
  share slides with friends.

  But we've all seen terrible font problems in important
  presentations, mostly due to incompatibilities between
  Windows and Mac versions of Powerpoint: fonts should be, but aren't,
  embedded in the presentation.

  And insertion of figures requires tedious copy-paste, usually followed
  by manual resizing and adjustment of figure placement.
  And if the figures are revised (because the data changed or some
  mistake was found in the analysis), we'll have to repeat all of
  that.
}
\end{frame}


\begin{frame}[c]{\LaTeX\/ Beamer package}

\figh{Figs/Copenhagen-default-default-01.png}{0.75}

\note{
  Until recently, I'd been making \LaTeX\/ slides using the {\tt
  article} document class, just revising the page size and make the
  fonts big.

  The Beamer package for \LaTeX\/ is easier, but I was turned off
  by the standard slides that people were producing with Beamer, such
  as the one shown: far too much junk on the screen, and on every
  single slide.

  You can get rid of all of that. All of the slides I'm making
  for this course are produced with Beamer.

  There's good facility for adding simple animations (progressively
  showing or hiding different elements on the slide).

  But you {\nhilit are} writing \LaTeX, so the coding can be a bit verbose.
}
\end{frame}



\begin{frame}[c,fragile]{Get rid of the junk}

\begin{lstlisting}
\usetheme{default}
\beamertemplatenavigationsymbolsempty

\definecolor{foreground}{RGB}{255,255,255}
\definecolor{background}{RGB}{24,24,24}
\definecolor{title}{RGB}{107,174,214}
\definecolor{subtitle}{RGB}{102,255,204}
\definecolor{hilit}{RGB}{102,255,204}
\definecolor{lolit}{RGB}{155,155,155}

\setbeamercolor{titlelike}{fg=title}
\setbeamercolor{subtitle}{fg=subtitle}
\setbeamercolor{institute}{fg=lolit}
\setbeamercolor{normal text}{fg=foreground,bg=background}
\setbeamercolor{item}{fg=foreground} % color of bullets
\setbeamercolor{subitem}{fg=lolit}
\setbeamercolor{itemize/enumerate subbody}{fg=lolit}
\setbeamertemplate{itemize subitem}{{\textendash}}
\setbeamerfont{itemize/enumerate subbody}{size=\footnotesize}
\setbeamerfont{itemize/enumerate subitem}{size=\footnotesize}

\newcommand{\hilit}{\color{hilit}}
\newcommand{\lolit}{\color{lolit}}
\end{lstlisting}

\note{
  The first thing to do is to get rid of all of the junk.

  Also, I prefer light text on a dark background.

  The tricky part is that Beamer has special names for
  everything.

  It would be best if I created a new theme, but I don't want to take
  the time to figure that out.
}
\end{frame}

\begin{frame}[c,fragile]{Also, slide numbers and fonts}

\begin{lstlisting}
% slide number
\setbeamertemplate{footline}{%
 \raisebox{5pt}{\makebox[\paperwidth]{\hfill\makebox[20pt]{\lolit
  \scriptsize\insertframenumber}}}\hspace*{5pt}}

% font
\usepackage{fontspec}
% http://www.gust.org.pl/projects/e-foundry/tex-gyre/
%      ...   heros/qhv2.004otf.zip
\setsansfont
  [ ExternalLocation = ../fonts/ ,
    UprightFont = *-regular ,
    BoldFont = *-bold ,
    ItalicFont = *-italic ,
    BoldItalicFont = *-bolditalic ]{texgyreheros}
% Palatino for notes
\setbeamerfont{note page}{family*=pplx,size=\footnotesize}
\end{lstlisting}

\note{
  I also want the slide number in the bottom-right, and I want a
  different font: something a bit more blocky, which I think is easier
  to read on the screen.
}
\end{frame}



\begin{frame}[c,fragile]{Title slide}

\begin{lstlisting}
\title{Put title here}
\subtitle{And maybe a subtitle}
\author{Author name}
\institute{Biostatistics \& Medical Informatics,
   UW{\textendash}Madison}
\date{\tt \scriptsize biostat.wisc.edu/{\textasciitilde}kbroman}

\begin{document}

{
\setbeamertemplate{footline}{} % no slide number here
\frame{
  \titlepage

\note{
  Summary of the talk, as a note.
}
} }
\end{lstlisting}

\note{
  The title slide is created with {\tt {\textbackslash}titlepage},
  having first defined {\tt {\textbackslash}title},
  {\tt {\textbackslash}author}, etc.

  The extra curly braces are to get the ``no slide number'' to apply
  just to the title slide. You can put notes on slides and then make a
  version that has the slide above the notes. See what I do with the
  slides for this course, or ask me for help.
}
\end{frame}


\begin{frame}<handout:0>[c,fragile]{Typical slide}

\begin{lstlisting}
\begin{frame}{Title of slide}

\bbi
 \item Bullet 1
 \item Bullet 2
 \item Bullet 3
\ei

\note{
  Put a note here
}
 \end{frame} % delete the leading space
\end{lstlisting}

\end{frame}


\begin{frame}[c,fragile]{Typical slide}
\addtocounter{framenumber}{-1}

\begin{lstlisting}
\begin{frame}{Title of slide}

\vspace{24pt} \begin{itemize} \itemsep8pt
 \item Bullet 1
 \item Bullet 2
 \item Bullet 3
\end{itemize}

\note{
  Put a note here
}
 \end{frame} % delete the leading space
\end{lstlisting}

\note{
  A typical slide is set between {\tt {\textbackslash}begin\{frame\}\{title\}}
  and {\tt {\textbackslash}end\{frame\}}.

  You get bullet points with the {\tt itemize} environment. I'll mess
  around a bit with {\tt {\textbackslash}vspace} and
  {\tt {\textbackslash}itemsep}. And I'll create shortcuts with
  {\tt {\textbackslash}newcommand} for these.

  I couldn't figure out how to get {\tt {\textbackslash}end\{frame\}}
  to appear in a slide, but adding a leading space worked. In
  practice, {\nhilit don't} include the space.
}
\end{frame}



\begin{frame}<handout:0>[c,fragile]{Slide with a figure}

\begin{lstlisting}
\begin{frame}{Title of slide}

\figh{Figs/a_figure.png}{0.75}


\note{
  Put a note here
}
 \end{frame} % delete the leading space
\end{lstlisting}

\end{frame}



\begin{frame}[c,fragile]{Slide with a figure}
\addtocounter{framenumber}{-1}

\begin{lstlisting}
\begin{frame}{Title of slide}

\centerline{\includegraphics[height=0.75\textheight]{%
            Figs/a_figure.png}}

\note{
  Put a note here
}
 \end{frame} % delete the leading space
\end{lstlisting}

\note{
  I'd typically generate figures externally and include them with
  {\tt {\textbackslash}includegraphics}. I got a good macro from Sam
  Younkin to simplify this.

}
\end{frame}



\begin{frame}[c,fragile]{Figures with KnitR}

\begin{lstlisting}
<<knitr_options, echo=FALSE>>=
opts_chunk$set(echo=FALSE, fig.height=7, fig.width=10)
change_colors <-
function(bg=rgb(24,24,24, maxColorValue=255), fg="white")
  par(bg=bg, fg=fg, col=fg, col.axis=fg, col.lab=fg,
      col.main=fg, col.sub=fg)
@

<<pdf_figure>>=
change_colors()
par(las=1)
n <- 100
x <- rnorm(n)
y <- 2*x + rnorm(n)
plot(x, y, pch=16, col="slateblue")
@
\end{lstlisting}

\note{
  You could use a knitr code chunk, in the same way we discussed
  with manuscripts, in the last lecture.
}
\end{frame}


\begin{frame}[c,fragile]{Figures with KnitR}

\begin{lstlisting}
% << >>= all on one line!
<<png_figure, dev="png", fig.align="center",
  dev.args=list(pointsize=30),
  fig.height=15, fig.width=15, out.height="0.75\\textheight",
  out.width="0.75\\textheight">>=
change_colors(bg=rgb(32,32,32,maxColorValue=255))
par(las=1)
n <- 251
x <- y <- seq(-pi, pi, len=n)
z <- matrix(ncol=n, nrow=n)
for(i in seq(along=x))
  for(j in seq(along=y))
    z[i,j] <- sin(x[i]) + cos(y[j])
image(x,y,z)
@
\end{lstlisting}

\note{
  To create a PNG figure (which can give much smaller file sizes for
  things like an image or a dense scatterplot), use the chunk option
  {\tt dev="png"}.

  For some reason, RGB colors don't match well between PNG files and
  the PDF, so I have to much about to get the background of the PNG to
  match the background on the slides.

  It's also a bit of work to get the resolution and text size just
  right.

  I split the initial line defining the code chunk across multiple
  lines here, so it could all be seen, but in practice the whole
  {\tt << >>=} bit needs to be on one line.
}
\end{frame}


\begin{frame}[c,fragile]{Slides with notes}

\begin{lstlisting}
\documentclass[12pt,t]{beamer}
\setbeameroption{hide notes}
\setbeamertemplate{note page}[plain]
\end{lstlisting}

\bigskip

\begin{lstlisting}
\documentclass[12pt,t,handout]{beamer}
\setbeameroption{show notes}
\setbeamertemplate{note page}[plain]
\def\notescolors{1}
\end{lstlisting}

\bigskip

\begin{lstlisting}
\ifx\notescolors\undefined % slides
  \definecolor{foreground}{RGB}{255,255,255}
  \definecolor{background}{RGB}{24,24,24}
\else % notes
  \definecolor{background}{RGB}{255,255,255}
  \definecolor{foreground}{RGB}{24,24,24}
\fi
\end{lstlisting}

\note{
  To create a version of your slides with notes, include
  {\tt {\textbackslash}note\{ \}} on every slide.

  I then include the code in the top box in the slide version, the
  middle box in the note version, and the stuff at the bottom in both.
  The bit at the bottom selects colors to be light text on a dark
  background in the slides and dark text on a light
  background in the notes version.

  I then use {\tt pdfnup} (part of PDFjam) to make 2-up pages (slides
  at the top, notes at the bottom). The only problem with {\tt pdfnup}
  is that it strips off all of the hyperlinks.
}
\end{frame}


\begin{frame}[c,fragile]{Simple animations}

\begin{lstlisting}
\begin{frame}{Bullets entering one at a time}

\bbi
\item Bullet 1
\onslide<2->{\item Bullet 2}
\onslide<3->{\item Bullet 3}
\onslide<4->{\item Bullet 4}
\ei

\note{
  Do this sparingly.
}
 \end{frame} % delete the leading space
\end{lstlisting}

\note{
  It's easy to add a bit of animation, such as with bullets appearing
  one by one. Use {\tt {\textbackslash}onslide} or
  {\tt {\textbackslash}only}.

  Here, the bullets will appear one at a time.

  Beamer just expands the PDF, with this slide becoming multiple
  pages.
}
\end{frame}




\begin{frame}[c,fragile]{Simple animations}

\begin{lstlisting}
\begin{frame}{Bullets entering one at a time}

\bbi
\item {\lolit \only<1>{\color{foreground}} Bullet 1}
\item {\lolit \only<2>{\color{foreground}} Bullet 2}
\item {\lolit \only<3>{\color{foreground}} Bullet 3}
\item {\lolit \only<4>{\color{foreground}} Bullet 4}
\ei

\note{
  Do this sparingly.
}
 \end{frame} % delete the leading space
\end{lstlisting}

\note{
  In this version, the bullets will go from dim to bright, one at a
  time.
}
\end{frame}



\begin{frame}<handout:0>[c]{Slidify and R Markdown}

\figh{Figs/slidify.png}{0.75}

\end{frame}


\begin{frame}[c,fragile]{Slidify and R Markdown}
\addtocounter{framenumber}{-1}

\begin{lstlisting}
## Slide title

- Bullet 1
- Bullet 2
- Bullet 3
- Bullet 4

---

## A figure

```{r a_figure, echo=FALSE, fig.align="center"}
par(las=1)
n <- 100
x <- rnorm(n)
y <- 2*x + rnorm(n)
plot(x, y, pch=16, col="slateblue")
```
\end{lstlisting}

\note{
  Slidify makes it super easy to create html-based slides with R
  Markdown. Three dashes separate slides, and two pound symbols
  (section heading) indicate the slide title.

  The chief advantage is that you can make nice slides with very little
  markup. And there are a ton of options, like having embedded
  quizzes.

  The disadvantage is that it's a bit harder to get fine control of
  the layout. And I've found it a bit risky to use html-based slides
  for a presentation. PDF is more trustworthy.

  In principle, you can use pandoc to convert the slides to PDF, but
  I've not been happy with the result. You could also print them from
  the browser, but I only got a good result with Safari. (Firefox
  included some links on the first page, and Chrome produced total
  garbage.)
}
\end{frame}


\begin{frame}[c,fragile]{Using slidify}

\begin{lstlisting}
library(devtools)
install_github("slidify", "ramnathv")
install_github("slidifyLibraries", "ramnathv")

library(slidify)
author("slidify_example")

# edit slidify_example/index.Rmd

slidify("index.Rmd")
\end{lstlisting}

\note{
  To use slidify, just download the slidify and slidifyLibraries
  packages, and use {\tt author()} to create the file to edit, and
  then {\tt slidify()} to compile the result.
}
\end{frame}


\begin{frame}[c,fragile]{YAML header}

\begin{lstlisting}
---
title       : Slidify example
subtitle    : Tools for reproducible research
author      : Karl Broman
job         : Biostatistics & Medical Informatics, UW-Madison
framework   : io2012        # {io2012, html5slides, shower, ...}
highlighter : highlight.js  # {highlight.js, prettify, highlight}
hitheme     : tomorrow      #
widgets     : [mathjax]     # {mathjax, quiz, bootstrap}
mode        : standalone    # {selfcontained, standalone, draft}
---
\end{lstlisting}

\note{
  There's a bit at the top of the file to define the slide title and
  layout.

  {\tt framework} defines the slide style. {\tt highlighter} is the
  method to give syntax highlighting. With {\tt mode} ``standalone,''
  some otherwise-external files are embedded in the html file.

  YAML is a ``human-readable data serialization format.''
  (Serialization means it can be easily transmitted over a network.)
  It's a well-defined way of describing potentially complex
  data objects.
}
\end{frame}



\begin{frame}[c,fragile]{Change the title slide colors}

\begin{lstlisting}
<style>
.title-slide {
  background-color: #EEE;
}

.title-slide hgroup > h1,
.title-slide hgroup > h2 {
  color: #005;
}
</style>
\end{lstlisting}

\note{
  The default colors for the title slide with framework {\tt io2012}
  are really terrible. Include a bit of CSS code in your {\tt .Rmd}
  file to fix that.

  There are a bunch of named colors in html, or you can use codes like
  {\tt \#005;} or {\tt \#000055;} for RGB (R=00, G=00, B=55).
}
\end{frame}


\begin{frame}{Summary}

\bbi
\item Use LaTeX/Beamer or Slidify to create reproducible slides.
\item Use LaTeX/Beamer to create reproducible posters.
\item Include KnitR code chunks to create figures directly.
\item Or keep the code for figures separate (which I tend to do).
\ei

\note{
  To make reproducible slides/posters, you need to dump PowerPoint.

}
\end{frame}




\end{document}
